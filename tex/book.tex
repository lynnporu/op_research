\documentclass[a5paper,10pt]{book}
\usepackage[T2A]{fontenc}
\usepackage[utf8]{inputenc}
\usepackage[english, ukrainian]{babel}

\usepackage{subfiles}

\usepackage[default,scale=.90]{opensans}
\usepackage{newtxmath}

\usepackage{amssymb}
\usepackage{amsmath}

\usepackage{geometry}
\geometry{left   = 1.5cm}
\geometry{right  = 2cm}
\geometry{top    = 2cm}
\geometry{bottom = 1.5cm}

\usepackage{indentfirst}
\renewcommand {\baselinestretch} {1.5}
\setlength\parindent{1cm}

\newcommand\nonumchapter[1]{
 \chapter*{#1}
 \addcontentsline{toc}{section}{#1}
}

\begin{document}

\tableofcontents

\subfile{parts/intro}

\chapter{Лінійне програмуванння}

\section{Графічний розв'язок}

\section{Симплекс-метод}

\section{Транспортна задача}

\section{Метод потенціалів}

\section{Цілочисельне програмування}

\chapter{Нелінійне програмування}

\chapter{Динамічне програмування}

\section{фівдт}

asdasd

\end{document}
