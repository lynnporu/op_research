\makeatletter
\def\input@path{{../}}
\makeatother
\documentclass[../book.tex]{subfiles}

\begin{document}

\chapter{Лінійне програмуванння}

Одночасно простою для розуміння та корисною на практиці є задача лінійного програмування (ЗПЛ). Задачу можна назвати лінійною тоді, коли цінова функція є функцією першого степеня~(\ref{eq:p2:linear-phi}), а область допустимих значень може бути описана лінійними рівностями або нерівностями~(\ref{eq:p2:linear-g-set}).

\begin{equation}
\begin{split}
 \varphi(\mathbf{X}) =
 \sum_{j=1}^n c_j x_j =
 \mathbf{c} \mathbf{X}^\mathrm{T},\;
 \mathbf{c} = \langle c_1, \ldots, c_n \rangle
\end{split}
\label{eq:p2:linear-phi}
\end{equation}

\begin{equation}
 G = \left\{
  \langle x_1, \ldots, x_n \rangle:
  \sum_{j=1}^n a_{ij} x_j\,
   \boxed{\leq, \geq \text{або} =}\,
  b_i
  \,\forall i = \overline{1, m}
 \right\}
 \label{eq:p2:linear-g-set}
\end{equation}

\section{Типові проблеми}

Приведені далі постановки задач є типовими. Вони отримали свою назву від проблеми, у вирішенні якої найкраще використовувати такі означення. Якщо їх проаналізувати, то можна отримати відпрацьований математичний апарат для розв'язання більш специфічних задач.

\subsection{Оптимальне використання ресурсів}

\textit{Для виробництва продукції $n$ видів потрібна певна кількість ресурсів $m$ видів, при чому фабрика має їх в обмеженій кількості. Відомо можливий прибуток від реалізації кожного з видів продукції. Потрібно визначити, скільки кілограм товару кожного виду варто виготовити, щоби отримати якнайбільший прибуток.}

Якщо між двома некерованими змінними існує відношення, то доцільно описати його у вигляді матриці. Тому нехай в $\mathbf{A}$ міститимуться елементи $a_{ij}$, кожен з яких означатиме витрату ресурсу $i$-ого виду на виготовлення продукції $j$-ого виду. Ще одна некерована змінна $\mathbf{b} = \langle b_1, \ldots, b_m \rangle$ позначає кількість відповідного ресурсу, доступного на фабриці. Прибуток від реалізації можна визначити вектором $\mathbf{c} = \langle c_1, \ldots, c_n \rangle$.

Керованою змінною є вектор $\mathbf{X} = \langle x_1, \ldots, x_n \rangle$, кожен елемент якого позначає кількість продукції, яку має виготовити фабрика -- ці дані варто змінювати для досягнення оптимальності.

Тоді умовою допустимості розв'язку буде перш за все виконання обмежень на доступну кількість ресурсів (кількість використаних обраховується з кількості виготовлених товарів) та деякі очевидні твердження (наприклад, те, що ця кількість має бути невід'ємною). Якби в умові задачі було вказано не \flqq{}кілограми\frqq{}, а одиниці, можна було би ще вимагати цілочисельності всіх $x_i$.

$$
 \mathbf{X} \in G \Leftarrow
 \left\{
  \begin{array}{l}
   \displaystyle \sum_{j=1}^n a_{ij} x_j \leq b_i\;
   \forall i = \overline{1, m} \\
   x_j \geq 0 \; \forall j = \overline{1, n}
  \end{array}
 \right\}
$$

Цінову функцію потрібно вводити так, щоби вона визначала прибутковість прийнятого рішення. Тоді оптимальним розв'язком буде такий, що гарантує досягнення її максимуму.

$$
 \varphi(\mathbf{X}) =
 \sum_{j=1}^n c_j x_j \rightarrow
 \max_{\mathbf{X} \in G}
$$

\end{document}
