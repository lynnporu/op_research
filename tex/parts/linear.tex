\makeatletter
\def\input@path{{../}}
\makeatother

\providecommand{\main}{..}
\documentclass[\main/book.tex]{subfiles}

\begin{document}

\chapter{Лінійне програмуванння}

Одночасно простим для розуміння та корисним на практиці є математичний апарат розв'язку задач лінійного програмування (ЛП), з яким читач познайомиться у цьому розділі.

\begin{itemize}
 \item У параграфі \sectionref{section:linear:typical} читач дізнається про деякі типові проблеми ЛП та отримає інтуїтивне розуміння того, як варто формулювати ОДР і цінову функцію.
 \item У параграфі \sectionref{section:linear:geometrical} описано графічний метод розв'язку проблем ЛП для $\mathbf{X} \in \mathbb{R}^2$.
 \item Параграф \sectionref{section:linear:simplex} описує симплекс-метод розв'язування ЛП для $n$-вимірних проблем ($\mathbf{X} \in \mathbb{R}^{\geq 2}$).
 \item У параграфі \sectionref{section:linear:integer} читач дізнається про принцип границь та галужень для розв'язування цілочисельної задачі ЛП.
 \item У параграфі \sectionref{section:linear:x_in_0-1} описано постановку та розв'язування ЛП задачі при $\mathbf{X} \in \{0,1\}^n$.
\end{itemize}

Задачу можна назвати лінійною тоді, коли цінова функція є функцією першого степеня~(\ref{eq:p2:linear-phi}), а область допустимих значень може бути описана лінійними рівностями або нерівностями~(\ref{eq:p2:linear-g-set}).

\begin{equation}
\begin{split}
 \varphi(\mathbf{X}) =
 \sum_{j=1}^n c_j x_j =
 \mathbf{c} \mathbf{X}^\top,\;
 \mathbf{c} = \langle c_1, \ldots, c_n \rangle
\end{split}
\label{eq:p2:linear-phi}
\end{equation}

\begin{equation}
 G = \left\{
  \langle x_1, \ldots, x_n \rangle:
  \sum_{j=1}^n a_{ij} x_j\,
   \boxed{\leq, \geq \text{або} =}\,
  b_i
  \,\forall i = \overline{1, m}
 \right\}
 \label{eq:p2:linear-g-set}
\end{equation}

\section{Типові проблеми}
\label{section:linear:typical}

Приведені далі постановки задач є типовими. Вони отримали свою назву від проблеми, у вирішенні якої найкраще використовувати такі означення. Якщо їх проаналізувати, то можна отримати відпрацьований математичний апарат для розв'язання більш специфічних задач.

\subsection{Максимізація прибутку}
\label{section:linear:phi_to_max}

\textit{Фабрика виробляє продукцію $n$ видів. Для неї потрібна певна кількість сировини $m$ видів, при чому її є в обмеженій кількості. Відомо можливий прибуток від реалізації кожного з видів продукції. Якщо існує якийсь оптимальний розподіл всієї сировини на виробництво певних видів продукції так, щоби отримати якнайбільший прибуток, то яким буде такий розподіл?}

Якщо між двома некерованими змінними існує відношення, то доцільно описати його у вигляді матриці. Тому нехай в $\mathbf{A}$ міститимуться елементи $a_{ij}$, кожен з яких означатиме витрату сировини $i$-ого виду на виготовлення продукції $j$-ого виду. Ще одна некерована змінна $\mathbf{b} = \langle b_1, \ldots, b_m \rangle$ позначатиме кількість відповідної сировини кожного виду, доступної на фабриці. Прибуток від реалізації можна визначити вектором $\mathbf{c} = \langle c_1, \ldots, c_n \rangle$.

Керованою змінною є вектор $\mathbf{X} = \langle x_1, \ldots, x_n \rangle$, кожен елемент якого позначає кількість продукції, яку має виготовити фабрика -- ці дані варто змінювати для досягнення оптимальності.

Тоді умовою допустимості розв'язку буде перш за все виконання обмежень на доступну кількість сировини, та деякі очевидні твердження -- наприклад, те, що ця кількість має бути невід'ємною (\ref{eq:p2:problem:phi_to_max:g_set}).

\begin{equation}
 \mathbf{X} \in G \Leftarrow
 \left\{
  \begin{array}{l}
   \displaystyle \sum_{j=1}^n a_{ij} x_j \leq b_i\;
   \forall i = \overline{1, m} \\
   x_j \geq 0 \; \forall j = \overline{1, n}
  \end{array}
 \right\}
 \label{eq:p2:problem:phi_to_max:g_set}
\end{equation}

\begin{problem}{problem:linear:integer_lp}
Чи можна поставити умову так, щоби знайдені $x_i$ обов'язково повинні були би бути цілими числами?
\end{problem}

В означенні (\ref{eq:p2:problem:phi_to_max:g_set}) фор\-му\-лю\-ва\-ння ${\sum_{j=1}^n a_{ij} x_j}$ позначає суму використаної сировини певного типу на виготовлення ресурсів всіх можливих видів. Відповідно \flqq{}$\leq b_i$\frqq{} позначає, що цією сировини можна використати не більше певної кількості. Умова $\forall i$ означає, що така нерівність повинна виконуватись для всіх видів сировини.

Цінову функцію потрібно вводити так, щоби вона визначала прибутковість прийнятого рішення. Тоді оптимальним розв'язком буде такий, що гарантує досягнення її максимуму (\ref{eq:p2:problem:phi_to_max:price}).

\begin{equation}
 \varphi(\mathbf{X}) =
 \sum_{j=1}^n c_j x_j \rightarrow
 \max_{\mathbf{X} \in G}
 \label{eq:p2:problem:phi_to_max:price}
\end{equation}

\begin{problem}{problem:linear:many_price_fn}
 Якщо існує така задача, для якої потрібно вводити кілька цінових функцій, і оптимізовувати їх одночасно, то якою буде умова такої задачі? Чи існує якийсь універсальний спосіб поєднання кількох оцінок оптимальності в одну?
\end{problem}

\subsection{Мінімізація витрат}

Розглянемо задачу, подібну тій, що поставлена у розділі~\ref{section:linear:phi_to_max}. \textit{Фабрика отримує сировину $n$ видів, з якої виготовляє товар за $m$ різними тех\-но\-ло\-гі\-я\-ми. Кожна з технологій потребує певну комбінацію сировини у різних кількостях. Разом з тим, кожна з технологій спричиняє в атмосферу викид певної кількості кілограм вуглецю. Для кожного виду сировини визначена доступна кількість, яку можна використати для виробництва. Потрібно оптимізувати виробництво так, щоби отримати якомога більше готової продукції, але водночас зменшити кількість шкідливих викидів.}

Що в цій задачі уже відомого? Найперше -- відношення технології та сировини, яку вона потребує. Запишемо це в матрицю $\mathbf{A}$, в якій кожен елемент $a_{ij}$ позначатиме витрату сировини $i$-ого типу за $j$-ою технологією. Викиди різних технологій можна зберігати у векторі $\mathbf{c} = \langle c_1, \ldots, c_m \rangle$, а кількість доступної сировини кожного виду -- як $\mathbf{b} = \langle b_1, \ldots, b_n \rangle$.

Що потрібно знайти? Потрібно знайти кількість товару, яка буде виготовлена за кожною з технологій. Тому позначимо її через вектор $\mathbf{X} = \langle x_1, \ldots, x_m \rangle$. При пошуку значення різних $x_j$ доведеться вирішити дилему: користуватись технологіями, які потребують меншу кількість сировини, але спричиняють більше викидів, чи навпаки?

Будь-який розв'язок $\mathbf{X}$ буде допустимим, якщо задовольнятиме дві умови: використано ресурсів не більше, ніж доступно взагалі, і жоден з $x_j$ не є від'ємним. Тому ознака допустимості розв'язку є такою самою, як і (\ref{eq:p2:problem:phi_to_max:g_set}) -- ми використовуємо ті самі назви змінних, тому формула не втрачає свого змісту. Тепер формулювання ${\sum_{j=1}^n a_{ij} x_j}$ позначатиме кількість сировини певного виду, використаної загалом, за якими технологіями б не відбувалось виробництво.

Допоки розв'язок допустимий, він використовуватиме лише дос\-туп\-ну кількість сировини. Однак різні розв'язки все ще спричиняють різний викид вуглецю, тому оптимальтність кожного з них визначатимемо саме за цим параметром. Цінова функція знову така сама, як і в (\ref{eq:p2:problem:phi_to_max:price}), проте цього разу її доцільно мінімізувати.

\begin{equation}
 \varphi(\mathbf{X}) =
 \mathbf{c} \mathbf{X}^\top
 \rightarrow \min_{\mathbf{X} \in G}
\end{equation}

\begin{note}
Тепер, після ознайомленнями з двома задачами, важливо розуміти, що не існує жодних правил, як варто складати формулювання ОДР чи цінової функції, так само як і неможливо передбачити всі можливі \flqq{}типові задачі\frqq{}. Формулювання будь-якої математичної моделі залежить винятково від області дослідження та винахідливості дослідника.
\end{note}

\subsection{План перевезень}
\label{subsection:linear:transportation_problem}

Досі ми розглядали задачі, де невідомою керованою змінною був вектор -- тобто, кількість чогось різних видів, або оптимальна конфігурація різних характеристик. Чи можливо тепер за допомогою лінійного програмування розв'язати якусь проблему, що встановлюватиме конфігурацію відношення? Тобто, цього разу невідоме $\mathbf{X}$ буде двовимірним -- матрицею.

\textit{Компанія має у своєму розпорядженні $n$ зерносховищ та $m$ постачальників зерна. Кожне зерносховище може зберігати певну кількість тон зерна, купленого у різних постачальників. З ними укладено контракт, за яким компанія зобов'язується викупити щонайменш певну кількість тон за раз. Відомі затрати на перевезення зерна від кожного постачальника до кожного сховища -- скажімо, час, відстань, пальне тощо. Потрібно визначити, скільки зерна між різними постачальниками та відповідними сховищами доцільно перевезти, щоби отримати якомога менші витрати, та перевезти якомога більше вантажу.}

\illustration
 {../imgs/linear/transport-graph.pdf}
 {Кожен з постачальників пропонує $\mathbf{a} = {\langle a_1, \ldots, a_m \rangle}$ тон, а кожне зі сховищ може зберігати до $\mathbf{b} = {\langle b_1, \ldots, b_n \rangle}$. Затрати на перевезення між ними задаються матрицею $\mathbf{C} \in \mathbb{R}^{m \times n}$.}
 {pic:linear:transport_problem}

Отже, умову проблеми можна зобразити у вигляді графа, що на Рис.~\ref{pic:linear:transport_problem}. Зрозуміло, що некерованими змінними є $\mathbf{a}$, $\mathbf{b}$ та $\mathbf{C}$. Тоді розв'язком буде матриця $\mathbf{X}$, кожен елемент $x_{ij}$ якої задаватиме \flqq{}оптимальну\frqq{} кількість тон, яку варто перевезти між $i$-им постачальником та $j$-им сховищем. Саме така конфігурація, що буде задана цією матрицею і гарантуватиме найменші витрати та якнайбільше заповнення сховищ.

\begin{note}
 Ця проблема є класичним випадком транспортної задачі, і для її розв'язку існують інші методи, про які можна дізнатись у наступних розділах. Попри це, розв'язок саме лінійним програмуванням є досить цікавим.
\end{note}

Тепер ОДР повинно мати такі обмеження:

\begin{itemize}
 \item потрібно вивезти весь товар, що надають постачальники;
 \item до кожного зі сховищ не можна привезти більше, ніж воно може вмістити.
\end{itemize}

Тоді умову допустимості розв'язку можна записати як~\ref{eq:linear:problem:matrix_to_matrix:g_set}.

\begin{equation}
 \mathbf{X} \in G \Leftarrow \left\{
 \begin{array}{l}
  \displaystyle
  \sum_{i=1}^m x_{ij} = a_i\; \forall j = \overline{1, n} \\
  \displaystyle
  \sum_{j=1}^n x_{ij} \leq b_j\; \forall i = \overline{1, m}
 \end{array}
 \right\}
 \label{eq:linear:problem:matrix_to_matrix:g_set}
\end{equation}

Щоби порахувати затрати при поточному плані перевезення, достатньо перемножити значення $c_{ij}$ з матриці затрат на значення $x_{ij}$. Оптимальним буде той розв'язок, у якому це значення найменше, тому вводимо цінову функцію (\ref{eq:linear:problem:matrix_to_matrix:price}), і вимагаємо її мінімізації.

\begin{equation}
 \varphi(\mathbf{X}) =
 \sum_{i=1}^m \sum_{j=1}^n
 c_{ij} x_{ij}
 \rightarrow \min_{\mathbf{X} \in G}
 \label{eq:linear:problem:matrix_to_matrix:price}
\end{equation}

\begin{problem}{problem:linear:transportation_problem}
 Нехай матриця $\mathbf{C}$ відображає не затрати на перевезення, а максимальну пропускну здатність маршруту. Скажімо, в день по дорозі між $i$-им постачальником та $j$-им сховищем можна перевезти не більше, ніж $c_{ij}$ тон вантажу. Потрібно знайти таке завантаження $\mathbf{X}$ різних доріг, щоби перевезти якомога більше вантажу. Допускається ситуація, коли не весь вантаж буде вивезено від постачальників. Як сформулювати ОДР та цінову функцію?
\end{problem}

\section{Геометричний сенс}
\label{section:linear:geometrical}

Двовимірні та (іноді) тривимірні випадки лінійного програмування можна розв'язувати графічно. Розуміння геометричних процесів, які відбуваються при відшукуванні розв'язку допоможуть при розв'язуванні багатовимірних задач.

\subsection{Двовимірний випадок}

Розгляньмо випадок, коли вектор $\mathbf{X}$ складається з двох координат $x_1$ та $x_2$ (тобто, ${\mathbf{X} \in \mathbb{R}^2}$). В ідеальному випадку множина ОДР ${G \in \mathbb{R}^2}$ утворює на площині деяку замкнену область, як показано на Рис.~\ref{pic:linear:example:g_set}. Зрозуміло, що коли обрати будь-яку точку з неї, то такі координати ${\langle x_1, x_2 \rangle}$ є допустимим розв'язком. Допустимих розв'язків існує безкінечна кількість (оскільки ми оперуємо множиною дійсних чисел, яка сама по собі є континуумом), але найоптимальнішим буде той, для якого цінова функція $\varphi(\mathbf{X})$ визначає якнайбільше число. Як можна знайти такий розв'язок графічно?

\illustration
 {../imgs/linear/2d-plain.pdf}
 {ОДР утворює опуклий багатокутник на площині.}
 {pic:linear:example:g_set}

Проаналізуймо цінову функцію. В загальному випадку вона матиме вигляд $\varphi = c_1 x_1 + c_2 x_2$, тобто утворюватиме площину в тривимірному просторі з координат $\langle x_1, x_2, \varphi \rangle$. Це можна уявити як на Рис.~\ref{pic:linear:example:phi_set}. Тоді допустимі розв'язки лежатимуть всередині проекції ОДР на площину цінової функції. З рисунку добре видно, в яку сторону вона зростає, тож тепер зрозуміло, що точка $\mathbf{X}_1$ є допустимим розв'язком з мінімально можливим $\varphi$, а $\mathbf{X}_5$ -- з максимальним $\varphi$. Тоді обидві точки будуть розв'язками задачі на мінімізацію та максимізацію цінової функції відповідно.

\illustration
 {../imgs/linear/2d-with-phi.pdf}
 {Погляд на ОДР з іншої перспективи. Пунктиром позначено проекцію $G$ на $\varphi$.}
 {pic:linear:example:phi_set}

Чи існує більш практичний спосіб, без потреби відмальовування тривимірних площин? Оскільки $\varphi$ -- лінійна функція, то за будь-яких $x_1$ та $x_2$ вона є неперервною, і зростає завжди однаково. Тобто, при будь-яких аргументах вектор градієнту завжди буде однакової довжини та спрямований в однаковому напрямку. Це означає, що на звичайному двовимірному малюнку достатньо позначити вектор $\overrightarrow{\mathrm{grad}}\,\varphi$ (як і показано на Рис.~\ref{pic:linear:example:g_set}). Тоді розв'язком задачі на максимізацію буде точка, що лежить якнайдалі у напрямку цього вектору. Аналогічно при мінімізації оптимальна точка розв'язку лежатиме у протилежному напрямку.

\begin{note}
 Продемонстрованим прикладом також можна пояснити випадок, коли цінова функція не є площиною, а має якийсь складніший вигляд, хоч це вже не буде задачею ЛП. Проекція $G$ на $\varphi$ сформує набір точок $\langle x_1, x_2, \varphi \rangle$, а оптимальним розв'язком буде та, що матиме найбільшу або найменшу координату $\varphi$.
\end{note}

\begin{problem}{problem:linear:gradient}
 Чи можна для розв'язування такої задачі використати метод градієнтного спуску? Яким чином його потрібно модифікувати? Які це дасть переваги? Які можуть виникнути проблеми?
\end{problem}

\begin{problem}{problem:linear:non_convex}
 Чи можна з допомогою системи лінійних рівнянь описати таку ОДР, яка утворюватиме неопуклий багатокутник? Як це працюватиме у вищих вимірах?
\end{problem}

\begin{conclusions}
\begin{itemize}
 \item оптимальний розв'язок може бути розташований в одній з кутових точок \footnote{(англ.) \textit{extreme point}};
 \item кутова точка є оптимальним розв'язком, якщо жодна з сусідніх точок не є оптимальнішою.
\end{itemize}
\label{conclusion:linear:optimal_solution}
\end{conclusions}

\illustration
 {../imgs/linear/brewery.pdf}
 {Запаси ресурсів, параметри ви\-роб\-ниц\-тва та прибутковість елю й пива.}
 {pic:linear:problem:brewery}

\subsection{Розв'язування задачі}

Розглянемо деяку специфічну задачу. \textit{Броварня виготовляє ель та пиво. Виробництво кожного з них вимагає певної комбінації кукурудзи, хмелю та ячмінного солоду на 1 умовну одиницю (бочку, партію тощо). Кількість ресурсів обмежена. Кожен продукт має свою ціну. Параметри виробництва показано на Рис.~\ref{pic:linear:problem:brewery}. Потрібно розрахувати оптимальну кількість вирбництва двох продуктів, щоби отримати якнайбільший прибуток.}

Некеровані змінні можна позначити вектором $\mathbf{a} = {\langle 220, 5, 540 \rangle}$ (кількість доступних ресурсів на складі), $\mathbf{b} = {\langle 213, 485 \rangle}$ (прибуток від кожного з товарів) та матрицею $\mathbf{A}$ (\ref{eq:linear:problem:brewery:a}), де $a_{ij}$ позначатиме витрату $i$-ого ресурсу на $j$-ий продукт. Вартість кожного з продуктів визначатиме вектор $\mathbf{c} = {\langle 213, 485 \rangle}$.

\begin{equation}
 \mathbf{A} = \begin{bmatrix}
               3 & 0.1 & 15 \\
               7 & 0.1 & 10
              \end{bmatrix}
 \label{eq:linear:problem:brewery:a}
\end{equation}

Тоді керованою змінною є вектор $\mathbf{X} = \langle x_1, x_2 \rangle$ координати якого позначають скільки бочок елю та пива потрібно виготовити відповідно. Якщо розв'язок лежить в ОДР, мусять виконуватись такі умови:

\begin{itemize}
 \item кількість використаних кукурудзи, хмелю та солоду не повинна перевищувати доступної;
 \item не можна виготовити від'ємну кількість продукції.
\end{itemize}

Кількість, використаної кукурудзи обраховується як ${3 x_1 + 7 x_2}$ (тобто, залежить від того, скільки виготовляється обох видів продукту). Оскільки маємо на неї обмеження, можна скласти нерівність ${3 x_1 + 7 x_2 \leq 220}$, що повинна виконуватись для допустимості розв'язку. Аналогічно складаються решту нерівностей, що утворюють ОДР задачі~(\ref{eq:linear:problem:brewery:g_set}).

\begin{equation}
 \mathbf{X} \in G \Leftarrow \left\{\begin{array}{l}
    3 x_1 + 7 x_2 \leq 220 \\
    0.1 x_1 + 0.1 x_2 \leq 5 \\
    15 x_1 + 10 x_2 \leq 540 \\
    x_1, x_2 \geq 0
 \end{array}\right\}
 \label{eq:linear:problem:brewery:g_set}
\end{equation}

Разом з тим оптимальність будь-якої комбінації $x_1$ та $x_2$ доцільно визначати за кількістю прибутку, що вона приносить. Тому цінова функція набуде вигляду~(\ref{eq:linear:problem:brewery:price}).

\begin{equation}
 \varphi(\mathbf{X}) =
 \mathbf{X} \mathbf{c}^\top =
 213 x_1 + 485 x_2
 \rightarrow \max_{\mathbf{X} \in G}
 \label{eq:linear:problem:brewery:price}
\end{equation}

На Рис.~\ref{pic:linear:problem:brewery:g_set} показано ОДР та градієнт цінової функції. З отриманого рисунку видно, що точка з ОДР, найдалі розташована у напрямку градієнта (а значить у напрямку зростання оптимальності), міститься в координатах $x_1 = 21.067$ та $x_2 = 22.4$. Це і є оптимальним розв'язком задачі.

\illustration
 {../imgs/linear/brewery-feasible.pdf}
 {ОДР задачі на двовимірній площині, що утворюється трьома нерівностями та обмеженнями на додатність $x_1$ і $x_2$.}
 {pic:linear:problem:brewery:g_set}
 
\begin{problem}{problem:linear:le_problem}
 Яким був би оптимальний розв'язок задачі, якби знак \flqq{}$\leq$\frqq{} в ОДР замінити на \flqq$<$\frqq?
\end{problem}

\begin{problem}{problem:linear:graphical_integer}
 Якби $x_1$ та $x_2$ визначали б не умовні одиниці, а вимірювалися б у бочках (тобто, тепер $\mathbf{X} \in \mathbb{Z}^2$), яким чином розв'язок такої задачі можна знайти графічно?
\end{problem}

Як видно, графічний спосіб розв'язку досить зручний і простий. Однак коли результат потрібно отримати з високою точністю, таке роз\-в'я\-зу\-ван\-ня стає дуже складним та непрактичним, не кажучи вже про те, що реальні задачі рідко коли бувають двовимірними. Проте потрібно віддати належне -- графічний спосіб ідеальний для унаочнювання даних.

\begin{problem}{problem:linear:dim_reduction}
 Чи можна якимось чином зменшити вимірність проблеми ${\mathbf{X} \in \mathbb{R}^{> 2}}$, щоби скористатись графічним способом розв'язування?
\end{problem}


\subsection{Крайнощі}

\twocolminipage
 {
  \nofigillustration
   {../imgs/linear/unbounded-g-set.pdf}
   {ОДР складається з однієї нерівності, тому є необмеженою зі сторони від'ємних $x_1$ та $x_2$. Тут можна розв'язати задачу на максимізацію, однак розв'язком мінімізації буде точка ${\langle -\infty, -\infty \rangle}$.}
   {pic:linear:unbounded_g_set}
 }{
  \nofigillustration
   {../imgs/linear/incountable-solutions.pdf}
   {Іноді неможливо визначити точку, що лежить найдалі за напрямком градієнта, оскільки всі точки на деякій лінії мають однакову оптимальність. В такому разі вони всі є оптимальними, і не існує єдиного $\mathbf{X}$.}
   {pic:linear:incountable_solutions}
 }

Описані раніше випадки, взагалі кажучи, є ідеальними, а із задачею нам просто пощастило.

Найперша проблема, яку іноді можна отримати -- ОДР не є замкненою, як показано на Рис.~\ref{pic:linear:unbounded_g_set}. В такому разі з тієї сторони, з якої ОДР необмежена, єдиного розв'язку не існує, позаяк до нього можна наближуватись безкінечно. Якщо така відповідь є допустимою, розв'язком можна вважати точку ${\langle \pm \infty, \pm \infty \rangle}$, однак найчастіше подібний результат просто не є адекватним.

Іноді оптимальних розв'язків існує \textit{безкінечна кількість}. Подібне формулювання може заплутати, і створити враження, ніби оптимальними є взагалі всі можливі розв'язки, хоча це не так. Така ситуація продемонстрована на Рис.~\ref{pic:linear:incountable_solutions}. У відповіді до цієї задачі доречно просто описати закон розрахунку оптимальних $x_1$ та $x_2$.

Може так трапитись, що нерівності, з яких складається ОДР, ніде не перетинаються одночасно, як показано на Рис.~\ref{pic:linear:full_unboundance}. В такому разі ОДР порожня, а отже жодного оптимального розв'язку не може існувати за визначенням.

\illustration
 {../imgs/linear/full-unboundance.pdf}
 {ОДР складається з 4 нерівностей, однак не існує жодної такої області, де вони всі перетиналися б одночасно. Тому $G = \varnothing$.}
 {pic:linear:full_unboundance}

\subsection{Тривимірний випадок}

Якщо у двовимірному просторі набір лінійних нерівностей може відтинати багатокутник, то у тривимірній області $G$ ми отримаємо багатогранник. Приклад такої фігури показано на Рис.~\ref{pic:linear:3d:g_set}. В цьому випадку читачу може бути складно уявити цінову функцію ${\varphi: \mathbb{R}^3 \rightarrow \mathbb{R}}$, позаяк вона є чотиривимірною, однак сенс залишається той самий: якщо перелік всіх точок фігури посортувати за тим, наскільки \flqq{}далеко\frqq{} вони розташовані в напрямку вектора-градієнта (а він тривимірний), то це одночасно буде перелік, посортований за зростанням оптимальності. Можна ще уявити це як зростання температури, яскравості чи будь-якої іншої простої для розуміння характеристики в певному напрямку.

\illustration
 {../imgs/linear/3d-g-set.pdf}
 {Оскільки така фігура має 7 граней, можна зробити висновок, що вона утворена 7-ома нерівностями. Будь-яка точка $\langle x_1, x_2, x_3 \rangle$ всередині неї є допустимим розв'язком. Зважаючи на напрямок вектора градієнту, точка $\mathbf{X}_2$ є розв'язком задачі на мінімізацію, а $\mathbf{X}_1$ -- на максимізацію цінової функції.}
 {pic:linear:3d:g_set}
 
\begin{note}
 Хоча отримана фігура досить проста на вигляд, малювати площини та визначати їхні перетини від руки насправді дуже складно. У вищих вимірностях проблема стає практично нерозв'язуваною. Тому хороший дослідник потребує більш точних аналітичних методів.
\end{note}

\section{Симплекс-метод}
\label{section:linear:simplex}

Далі читач дізнається про симплекс-метод -- інструмент, за допомогою якого можна аналітично розв'язувати багатовимірні задачі ЛП. З висновків на с.~\pageref{conclusion:linear:optimal_solution} може з'явитись така ідея:

\begin{itemize}
 \item якщо оптимальний розв'язок розташований в одній з кутових точок, потрібно отримати список всіх їхніх координат;
 \item опісля обраховуємо значення цінової функції $\varphi$ у всіх точках;
 \item обираємо точку з найбільшим $\varphi$, тобто найбільшою оптимальністю -- розв'язок знайдено.
\end{itemize}

\begin{problem}{problem:linear:infite_optimum}
 Якщо дійсно реалізовувати таку ідею -- як визначити випадок, коли весь відрізок між двома точками визначає оптимальні розв'язки?
\end{problem}

Однак насправді це не надто вдале рішення, бо задачі \flqq{}реального світу\frqq{} можуть мати тисячі різних рівнянь, тому часові та обчислювальні затрати на те, щоби визначити, які з них утворюють між собою перетини, та обчислення їхніх координат
, будуть просто неадекватними. Натомість симплекс-метод пропонує такий підхід:

\begin{itemize}
 \item починаємо обхід фігури з деякого початкового рішення (\textbf{опорного плану}\footnote{(англ.) \textit{basic feasible solution}}), яке точно є допустимим -- наприклад $\mathbf{X}_0 = {\langle 0, 0, \ldots \rangle}$;
 \item аналізуємо цінову функцію: якщо збільшення якоїсь зі змінних призведе до її зростання, то збільшуємо її, наскільки можливо;
 \item перераховуємо решту змінних (деякі з них можуть зменшитись), і отримуємо точку, яка лежить в одній із сусідніх кутових точок з більшою оптимальністю;
 \item перераховуємо коефіцієнти цінової функції так, щоби вона показувала, які зі змінних можна ще збільшувати;
 \item якщо зміна значень всіх змінних призведе до спадання $\varphi$, то найоптимальніший розв'язок вже знайдено.
\end{itemize}

Для реалізації цього алгоритму використовується апарат лінійної алгебри, за допомогою якого можна отримати формули для розрахунку розв'язку. Однак дуже часто їх залишають без пояснення, тож дослідник має лише інструкцію до виконання -- зовсім незрозумілу та дуже абстрактну. Щоби уникнути такої ситуації, потрібно спочатку зрозуміти деякі важливі концепти. Можливо, спершу вони видаватимуться непов'язаними один з одним, проте пізніше читач по-справжньому оцінить їхню елегантність.

\subsection{Важливі концепти}

\subsubsection{Люзові змінні}

Будь яку лінійну нерівність $f(x_1, \ldots) \leq b$ можна перетворити у рівність $f(x_1, \ldots) + s = b$. Тоді при ${s=0}$ ми отримаємо лінію, яка лежить на межі області нерівності. При збільшенні $s$ вона буде паралельно переноситись все далі від цієї межі, аж доки при $s=b$ не пройде через центр координат. Отже, при всіх $s \geq 0$ отримане рівняння описуватиме всі можливі точки $\langle x_1, x_2, \ldots \rangle$ всередині області нерівності. Це показано на Рис.~\ref{pic:linear:slack_inequality_leq} та Рис.~\ref{pic:linear:slack_inequality_geq}. Такі змінні $s$ називається \textbf{люзовими}\footnote{(англ.) \textit{slack variables} для нерівностей зі знаком~\flqq{}$\leq$\frqq{}, \textit{surplus variables} для нерівностей~\flqq{}$\geq$\frqq{}.}.

\begin{figure*}
 \centering
 \begin{minipage}[t]{.47\textwidth}
  \includegraphics{../imgs/linear/slack-inequality-leq.pdf}
  \caption{Нерівність $x_1 + x_2 \leq 4$ утворює область, межу якої можна описати рівнянням $x_1 + x_2 + s = 4$ при $s=0$. При $s=b=4$ ця лі\-нія проходить через центр координат.}
  \label{pic:linear:slack_inequality_leq}
 \end{minipage}\qquad
 \begin{minipage}[t]{.47\textwidth}
  \includegraphics{../imgs/linear/slack-inequality-geq.pdf}
  \caption{Для того, щоби отримати такий самий ефект для нерівностей \flqq{}$\geq$\frqq{}, змінну $s$ потрібно включити у рівняння прямої зі знаком~\flqq{}$-$\frqq{}.}
  \label{pic:linear:slack_inequality_geq}
 \end{minipage}
\end{figure*}

Проаналізуймо нерівність $f(x_1, x_2, \ldots) \leq b$, яким зазвичай задаються умови на ОДР задачі лінійного програмування. Її можна переписати так:
 
\[
 \begin{array}{lrlll}
  f(x_1, x_2, \ldots) &  \leq b &                         & \equiv &            \\
  f                   & + s = b & \forall s > 0           & \equiv & \text{(1)} \\
  s = b - f           &         & \forall x_1, x_2, \ldots &        & \text{(2)}
 \end{array}
\]

На кроці (1) ми ввели люзову змінну, а на кроці (2) -- перенесли $f$ в праву частину рівняння. Оскільки $b$ завжди позначає кількість доступного ресурсу, а $f$ -- функція, що обраховує кількість використаних (перемножуючи $x_1, x_2, \ldots$ на відповідні $a_{n1}, a_{n2}, \ldots$), то логічно, що $s$ можна інтерпретувати як \textit{кількість невикористаних ресурсів}.

\subsubsection{Система рівнянь з люзовими змінними}

Нехай маємо таку систему нерівностей:

\[
 \left\{
  \begin{array}{l}
   a_{11} x_1 + a_{12} x_2 \leq b_1 \\
   a_{21} x_1 + a_{22} x_2 \geq b_2
  \end{array}
 \right. .
\]

Введемо змінну $\delta_n$. Якщо $n$-на нерівність в системі має знак \flqq{}$\leq$\frqq{}, то $\delta_n=1$, а якщо навпаки -- то $\delta_n=-1$. Тепер систему нерівностей можна переписати як систему рівностей, а також отримати аналогічний запис у матрично-векторному-вигляді.

\begin{figure}[!h]
 \center
 \includegraphics{../imgs/linear/converting-to-matrix-vector.pdf}
\end{figure}

Ще коротший запис -- $\mathbf{A} \mathbf{X}^\top + \mathbf{\delta} \mathbf{s}^\top = \mathbf{b}^\top$. Аналогічно можна представити систему нерівностей будь-якої вимірності.

\begin{problem}{problem:linear:convex_statement}
 Нехай $\mathbf{\delta}$ -- одинична матриця. Який висновок можна зробити про систему нерівностей? Який вигляд матиме ОДР, якщо створити додаткову умову $\forall i: x_i \geq 0$?
\end{problem}

\begin{problem}{problem:linear:new_slack}
 Маємо систему з люзовими змінними за прикладом, що вище. Змінним $x_1$ та $x_2$ присвоєно деякі значення так, що вони визначають розв'язок системи. Зрозуміло, що при зміні їхнього значення, $s_1$ та $s_2$ потрібно перерахувати, щоби рівності знову справджуватись. Як вивести формулу для обрахунку нових люзових змінних?
\end{problem}

\begin{problem}{problem:linear:new_x1x2}
 Аналізуємо ту саму систему рівнянь. Нехай при значеннях $s_1 = s_2 = 0$ система має деякий розв'язок $\mathbf{X} = \langle x_1, x_2 \rangle$, як показано на рисунку нижче. Очевидно, що якщо змінити значення $s_1$, то розв'язок системи теж зміниться. Як знайти його?
 \center
 \includegraphics{../imgs/linear/simple-slack-solution.pdf}
\end{problem}

\subsubsection{Оцінка росту функції}
\label{section:linear:phi_increase_evaluation}

Нехай маємо цінову функцію у загальному вигляді $\varphi = c_1 x_1 + c_2 x_2$. Уявімо також, що на площині визначена деяка точка ${\langle x_1, x_2 \rangle}$. Дозволено збільшувати одну з її координат. Яку з них доцільно обрати таку, щоби значення $\varphi$ зростало якнайшвидше?

\flqq{}Найвигіднішу\frqq{} координату можна обрати як за допомогою частинного диференціювання, так і користуючись простою арифметичною логікою: у значення $\varphi$ найбільший вклад вносить така змінна $x_i$, біля якої стоїть найбільший коефіцієнт $c_i$. Приклад конкретної задачі показано на~Рис.~\ref{pic:linear:price_problem_1}.

\illustration
 {../imgs/linear/price-problem1.pdf}
 {Ізолініями позначено значення фукнції $\varphi = 2 x_1 + 5 x_2$. Видно, що збільшення координати $x_2$ дає більший зріст $\varphi$, ніж збільшення $x_1$ на те саме значення. Це стається тому, що $\dfrac{\partial\varphi}{\partial x_1} > \dfrac{\partial\varphi}{\partial x_2}$, або по-іншому: $c_1 > c_2$.}
 {pic:linear:price_problem_1}

\begin{problem}{problem:linear:steepest_point}
 З точки $\langle x_0, y_0 \rangle$ можна перейти до $\langle x_1, y_1 \rangle$ або $\langle x_2, y_2 \rangle$, а цінова функція задана як $\varphi = x c_1 + y c_2$. Чи можна за рисунком оцінити, перехід до якої точки призведе до найбільшого зростання $\varphi$? Чи можна також зробити це аналітично, не обраховуючи значення $\varphi$? Якщо можна, то як? Чи працюватиме це для багатовимірних просторів?
 \center
 \includegraphics{../imgs/linear/steepest-point.pdf}
\end{problem}

\subsubsection{Люзові змінні як базис}

\twocolminipage
 {
  \nofigillustration
   {../imgs/linear/slack-replace-x2.pdf}
   {Нові осі теж створюють двовимірний простір -- щоправда, трохи \flqq{}спотворений\frqq. Вздовж всієї лінії змінюється лише координата $x_1$, а $s_1$ залишається сталою.}
   {pic:linear:slack_replace_x2}
 }{
  \nofigillustration
   {../imgs/linear/slack-line-on-x1x2.pdf}
   {Той самий шлях у базисі ${\langle x_1, x_2 \rangle}$ вимагає зміни двох координат $x_1$ та $x_2$ одночасно.}
   {pic:linear:slack_line_on_x1x2}
 }

Погляньмо ще раз на Рис.~\ref{pic:linear:slack_inequality_leq}. Можна уявити, що $s$ змінюється по деякій діагональній осі. Спробуймо провести її на графіку, викинувши якусь іншу, наприклад, $x_2$, як на Рис.~\ref{pic:linear:slack_replace_x2}. Така дія називається \flqq{}переходом до нового базису\frqq. Використовуючи нові осі, можна представити будь-яку точку простору. Таку систему координат ми називаємо \flqq{}простором з базисом ${\langle x_2, s \rangle}$\frqq{} Якщо в ньому рухати якусь точку, змінюючи всього одну координату, то легко побачити, що аналогічний рух на графіку з базисом ${\langle x_1, x_2 \rangle}$ вимагатиме зміни двох координат одночасно, як на Рис.~\ref{pic:linear:slack_line_on_x1x2}.

Чому це може бути зручно? Справа в тім, що в новому базисі ${\langle x_1, s \rangle}$ моделювати рух по потрібній лінії досить просто -- ми ж змінювали лише одну змінну. Водночас, завжди є можливість повернутись назад, до звичного базису ${\langle x_1, x_2 \rangle}$, отримавши нові координати, які інакше довелося б розраховувати, наприклад, з рівняння прямої.

\begin{note}
  Операція заміни базису є тривіальним завданням в лінійній алгебрі. Нові координати можуть бути обчислені за допомогою спеціальних матриць переходу, або з системи рівнянь.
\end{note}

\begin{problem}{problem:linear:circular_space}
 Чи можна створити такий простір $O$, в якому рух по прямій перетворювався би в рух по колу в просторі $R$? Якими будуть формули переходу? Яку мінімальну вимірність повинні мати такі простори? Чи є таке перетворення лінійним?
\end{problem}

\illustration
 {../imgs/linear/slack-figure1.pdf}
 {Три рівняння при $s_{1,2,3}=0$ та осі $x_1$ і $x_2$ замикають область опуклого п'ятникутника. На перетинах усіх ліній позначено точки ${\mathbf{X}_0 \ldots \mathbf{X}_4}$. Для простоти зображення відповідні осі $s_1$, $s_2$ та $s_3$ не показано.}
 {pic:linear:slack_figure1}

Нехай ми маємо, наприклад, систему рівнянь таку, як показана на Рис.~\ref{pic:linear:slack_figure1}. Можна побачити, що перебуваючи в базисі ${\langle x_1, s_1 \rangle}$, і змінюючи лише координату $x_1$ при $s_1 = 0$ можна потрапити з точки $\mathbf{X}_1$ до $\mathbf{X}_2$. Для того, щоби здійснити наступний перехід до $\mathbf{X}_3$, зручно включити в базис $s_2$. Тепер він складатиметься з трьох змінних, і оскільки ми працюємо з двовимірним простором, одну з них можна \flqq{}викинути\frqq{}. Зазвичай \flqq{}викидають\frqq{} ту змінну, значення якої ми вже змінили раніше.

\begin{note}
 Позначатимемо надалі перехід від однієї точки до іншої знаком \flqq{}$\overset{\langle \cdot, \cdot \rangle}{\rightarrow}$\frqq -- стрілкою, над якою позначено базис, у якому зручно моделювати таке переміщення.
\end{note}

Отже, рух вздовж фігури, що окреслена заданою системою рівнянь, матиме такий вигляд:
$
\mathbf{X}_0
\overset{\langle x_1, x_2 \rangle}{\rightarrow}
\mathbf{X}_1
\overset{\langle x_1, s_1 \rangle}{\rightarrow}
\mathbf{X}_2
\overset{\langle s_2, s_1 \rangle}{\rightarrow}
\mathbf{X}_3
\overset{\langle s_2, s_3 \rangle}{\rightarrow}
\mathbf{X}_4
\overset{\langle x_2, s_3 \rangle}{\rightarrow}
\mathbf{X}_0
$. Якщо уважно прослідкувати за ним, то можна помітити, що змінна, яка входить в новий базис -- це завжди та, що визначає лінію, яка перетинається з траєкторією руху найперша. До прикладу: якщо ми рухаємось вздовж осі $x_2$ в напрямку зростання, то спочатку ми перетинаємо лінію $s_1 = 0$, а тоді -- $s_2 = 0$. Отже в новий базис замість $x_2$ (яку ми вже змінювали раніше) увійде $s_1$.

Якщо повторити цю проблему у тривимірному, або в будь-якому $n$-вимірному просторі, то базис переходу міститиме три, або $n$ змінних відповідно. Такі процеси вже неможливо візуалізувати, саме тому так важливо створити хорошу аналітичну модель.

\subsubsection{Оцінка росту функції в новому базисі}

Зрозуміло, що в процесі ходу алгоритму ми переходимо від якогось $\mathbf{X}_a$ до $\mathbf{X}_b$ не просто так, а лише тому, що $\varphi(\mathbf{X}_b) > \varphi(\mathbf{X}_a)$, і ці дві точки є сусідніми. З розділу~\ref{section:linear:phi_increase_evaluation} читач вже знає, що для оцінки того, яку змінну варто змінювати, потрібно просто порівняти коефіцієнти $c_i$. Однак що робити, коли цінова функція задана як $\varphi(x_1, x_2)$, але ми перебуваємо в базисі, наприклад, $\langle x_1, s_1 \rangle$? Маючи систему рівностей таку саму, як приведено на Рис.~\ref{pic:linear:slack_figure1}, можна виразити $x_2$ з першого рівняння. Тоді отримуємо:

\[
  \begin{array}{ll}
    a_{11} x_1 + a_{12} x_2 + s_1 = b_1          & \equiv      \\
    x_2 = \dfrac{b_1 - s_1 - a_{11} x_1}{a_{12}} & \Rightarrow \\
    \varphi(x_1, x_2) = c_1 x_1 + c_2 x_2        & \Rightarrow \\
    \varphi(x_1, s_1) = c_1 x_1 + c_2 \left(
      \dfrac{b_1 - s_1 - a_{11} x_1}{a_{12}}
    \right)                                      & \equiv      \\
    \varphi(x_1, s_1) =
    c_2 a_{12} c_1 x_1 - c_2 b_1 - c_2 s_1 - c_2 a_{11} x_1
  \end{array}
\]

Як видно, при переході до нового базису ми отримали лінійну функцію. Тому тепер для оцінки росту достатньо лише порівняти коефіцієнти при відповідних змінних. Аналогічний процес відбуватиметься при переході між будь-якими кутовими точками фігури.

\subsection{Аналітичне розв'язування}

\subsection{Симплекс-таблиці}


\begin{problem}{problem:linear:new_basis}
 Маємо постановку ЛП з $n$ змінними та $m$ рівняннями, тобто $\mathbf{A} \in \mathbb{R}^{n \times m}$. Нехай поточний базис -- $\langle s_1, s_2 \ldots, s_n \rangle$. З нього виходить змінна $s_7$, а натомість входить $x_5$. Якщо пронумерувати всі рівняння ОДР, то з якого з них потрібно виразити $x_5$, щоби потім підставити формулювання з $s_7$ у цінову функцію? А якби з базису виходила змінна $s_{10}$, і замість неї входила б $x_{28}$? Чи можна вивести якусь загальну формулу нових коефіцієнтів при змінних у ціновій функції з потрібним базисом?
\end{problem}

\begin{equation}
 \begin{array}{rl}
  \text{оптимізувати}  & \varphi = 2 x_1 + 3 x_2 \rightarrow \max \\
  \text{з обмеженнями} & 2 x_1 +   x_2 \leq 4; \\
                       &   x_1 + 3 x_2 \leq 4.
 \end{array}
 \label{eq:linear:simplex:example_task}
\end{equation}

\begin{figure}
 \centering
 \includegraphics{../imgs/linear/simplex-tableau-1.pdf}
\end{figure}

\begin{figure}
 \centering
 \includegraphics{../imgs/linear/simplex-tableau-expl-1.pdf}
\end{figure}
\section{Цілочисельне лінійне програмування}
\label{section:linear:integer}

\section{Проблеми оптимального вибору}
\label{section:linear:x_in_0-1}

Припустимо таку ситуацію, коли ми маємо набір різних рішень, з яких складається розв'язок проблеми. Наприклад, \textit{складне обчислення можна розпаралелити на $n$ процесорів, кожен з яких має свою продуктивність операцій/с та потужність у ватах. Скільки і які процесори потрібно увімкнути, щоби продуктивності вистачило на виконання деякого обчислення у вказаний час, але спожити щонайменше енергії?} Тоді набір всіх можливих рішень

\[
\begin{split}
P = \{
 \text{увімкнути процесор 1},
 \text{увімкнути процесор 2},
 \ldots, \\
 \text{увімкнути процесор $n$}
\}
\end{split}
.
\]

Відповідно, будь-який розв'язок задачі буде лише підмножиною $P$. Однак такими формулюваннями оперувати дуже не дуже зручно, тому краще ввести характеристичний вектор $\mathbf{X}$, кожен елемент $x_i$ якого позначатиме, чи прийняте деяке $i$-те можливе рішення. Так, якщо $x_i=0$, то $i$-ий процесор варто вимкнути, але якщо $x_i=1$ -- то навпаки.

\begin{note}
 Формально такий випадок можна записати як $\mathbf{X} \in \{0, 1\}^{n}$, або простіше -- $\forall i: x_i \in \{0, 1\}$.
\end{note}

\clearpage
\blindsection{Розв'язування проблем}

\subsection*{Проблема №\arabic{chapter}.\ref{problem:linear:integer_lp}}

Звичайно, на світі існує безліч ресурсів, які не вийде представити дробами: не можна відправити на роботу двох з половиною робітників, продати половину телевізора абощо. Тоді до умови ОДР додається ще одна: $\forall i: x_i \in \mathbb{Z}$. Пошук такого розв'язку може бути набагато складнішим, ніж коли $\forall i: x_i \in \mathbb{R}$.

\subsection*{Проблема №\arabic{chapter}.\ref{problem:linear:many_price_fn}}

Кілька цінових функцій може з'явитись тоді, коли розв'язок описується кількома характеристиками. Звичайно, якщо вони якимось чином залежать одна від одної, для спрощення задачі краще обійтись однією (або будь-яким іншим мінімальним числом). Наприклад, між відстанню, яку проходить автомобіль, та кількістю спожитого палива існує пряма залежність, тому при мінімізації чогось одного інша характеристика також зменшуватиметься вже без будь-якого впливу.

Проте якщо існують деякі $\varphi_1 \ldots \varphi_i$, і вони не є пов'язаними, то їх можна, наприклад, просто додати: $\varphi = \sum_i \varphi_i$. В такому разі мінімізація або максимізація $\varphi$ означатиме аналогічну зміну всіх $\varphi_i$.

Припустимо, що якусь $\varphi_n$ потрібно мінімізовувати, а решту $\varphi_i \; \forall {i \neq n}$ -- максимізовувати. Тоді $(-\varphi_n) \rightarrow \max \Rightarrow \varphi_n \rightarrow \min$. В такому разі сумарну оцінку можна сформувати як $\varphi = \varphi_1 + \ldots + \varphi_i - \varphi_n$ -- тоді умова $\varphi \rightarrow \max$ гарантуватиме правильне знаходження розв'язку.

Коли це може бути корисним? Припустимо, продукція різних видів $x_i$ по двох магазинах, де продається за різними цінами $a_i$ та $b_i$ відповідно. В такому разі \flqq{}вигода\frqq{} прийнятого рішення характеризуватиметься двома оцінками $\varphi_a = \mathbf{X} \mathbf{a}^\top$ та $\varphi_b = \mathbf{X} \mathbf{b}^\top$. Тоді максимізація функції $\varphi = \varphi_a + \varphi_b$ забезпечуватиме правильне знаходження розв'язку.

\subsection*{Проблема №\arabic{chapter}.\ref{problem:linear:transportation_problem}}

Оскільки за умовою потрібно \flqq{}перевезти якомога більше вантажу\frqq{}, то задля оцінки прийнятого рішення достатньо просто додати всі елементи матриці $\mathbf{X}$:

\[
 \varphi(\mathbf{X}) = \sum_{i=1}^{m} \sum_{j=1}^{n} x_{ij}
\]

До обмежень, приведених у розділі \ref{subsection:linear:transportation_problem}, потрібно додати ще одне, яке запобігатиме перевантаженню доріг. Тоді ОДР матиме такий вигляд:

\[
 \mathbf{X} \in G \Leftarrow \left\{
  \begin{array}{l}
   \vdots \\
   \forall i, j: x_{ij} \leq c_{ij}
  \end{array}
 \right.
\]

\subsection*{Проблема №\arabic{chapter}.\ref{problem:linear:gradient}}

Теоретично, з допомогою градієнтного спуску можна знайти екстремум будь-якої диференційовної функції, якщо область допустимих розв'язків є неперервна. Позначимо $\mathbf{X}_\alpha = \mathbf{X} + \alpha \, \overrightarrow{\textrm{grad}} \, \varphi \Bigr|_\mathbf{X}$ -- точка, зсунута від даної $\mathbf{X}$ у напрямку вектора градієнту функції $\varphi$ на деякий коефіцієнт $\alpha$. Тоді можна стверджувати, що алгоритм не зможе знайти правильну відповідь, якщо $\exists \widetilde{\mathbf{X}} \in G \, \exists \alpha > 0: \varphi(\mathbf{X}_\alpha) < \varphi(\widetilde{\mathbf{X}}) \;\&\; \mathbf{X}_\alpha \notin G$. Неформально кажучи, подальші кроки у напрямку градієнту виводять нас із ОДР, хоча якщо трохи повернути вектор градієнту, то значення $\varphi$ -- хоч і не так стрімко -- можна збільшити ще. Таку ситуацію можна уявити так, як показано на Рис.~\ref{pic:linear:problem:gradient_out_of_g}.

\illustration
 {../imgs/linear/gradient-out-of-g.pdf}
 {Червона область -- це набір всіх точок $\langle x_1, x_2, \varphi(x_1, x_2)\rangle$ таких, що $\langle x_1, x_2 \rangle \in G$. Якщо слідувати за вектором градієнта, найбільш можлива оптимальність, яку можна досягнути -- $\varphi(\mathbf{X}_\alpha)=1.2$. Насправді на ОДР існує точка $\widetilde{\mathbf{X}}$, яка має ще більшу оптимальність, однак вектор градієнту ніколи на неї не вкаже.}
 {pic:linear:problem:gradient_out_of_g}

Для того, щоби вирішити цю проблему, можна придумати якийсь алгоритм повороту вектора тощо. Втім, подальші дії виходять за межі звичайної вправи, і можуть перетворитись в актуальне дослідження.

\subsection*{Проблема №\arabic{chapter}.\ref{problem:linear:non_convex}}

Опуклим можна вважати такий многокутник, який не містить самоперетинів, а також кожен з його внутрішніх кутів є меншим за $180^\circ$. Спробуймо утворити ОДР за допомогою двох нерівностей так, як показано на Рис.~\ref{pic:linear:problem:convex}.

\illustration
 {../imgs/linear/convex-set.pdf}
 {Область ОДР -- перетин нерівностей ${A: a_{11} x + a_{12} y \leq b_1}$ та ${B: a_{21} x + a_{22} y \leq b_2}$.}
 {pic:linear:problem:convex}

З рисунку видно, що навіть якщо збільшувати кут $\gamma$ між межами двох нерівностей так, що він стане більшим за $180^\circ$, область ОДР все одно залишатиметься опуклою. Змінюючи знаки нерівностей, узагальнюючи такі висновки на решту кутових точок ОДР та багатовимірні простори, наполегливий дослідник зможе дійти до аналітичного доведення такого явища.

\subsection*{Проблема №\arabic{chapter}.\ref{problem:linear:le_problem}}

Для розуміння проблеми найпростіше розглянути її в одновимірному випадку. Нехай ОДР визначена як

\[
 G = \left\{
  \begin{array}{l}
   x \geq a_1 \\
   x \leq a_2
  \end{array}
 \right.,
\]

тоді зрозуміло, що $x$ міститься у закритому відрізку $[a_1, a_2]$, і розв'язком задачі буде одна з його меж. Якщо ж відрізок відкритий, і нам потрібно розв'язати задачу, наприклад, на максимізацію то розв'язком буде деяка границя $x = \displaystyle \lim_{a \to a_1} a$, яку можна обчислити з якою завгодно точністю $\varepsilon$. Припустимо, що $a_1 = 5$, тоді:

\begin{figure}[!h]
 \center
 \begin{tabular}{ll}
  $x = 4.9$,     & $\varepsilon = 10^{-1}$; \\
  $x = 4.999$,   & $\varepsilon = 10^{-3}$; \\
  $x = 4.99999$, & $\varepsilon = 10^{-5}$.
 \end{tabular}
\end{figure}

Розглянемо також двовимірний випадок. Нехай ОДР задана як

\[
 G = \left\{
  \begin{array}{l}
   a_{11} x + a_{12} y < b_1 \\
   a_{21} x + a_{22} y < b_2
  \end{array}
 \right..
\]

Зрозуміло, що на двовимірній площині до точки розв'язку можна наближатись із якої завгодно сторони, тому розумним рішенням буде знайти деяке гранично допустиме значення $x$ та $y$ окремо\footnote{Так само як замість того, щоби говорити про диференціал багатозмінної функції ми аналізуємо натомість частинні похідні.}. Розв'язавши систему відносно $x$ та $y$ отримаємо:

\[
 \begin{array}{ll}
 G = \left\{
  \begin{array}{l}
   x < \dfrac{b_1 - a_{12} y}{a_{11}} \\
   x < \dfrac{b_2 - a_{22} y}{a_{21}}
  \end{array}
 \right.;
 &
 G = \left\{
  \begin{array}{l}
   y < \dfrac{b_1 - a_{11} x}{a_{12}} \\
   y < \dfrac{b_2 - a_{21} x}{a_{22}}
  \end{array}
 \right..
 \end{array}
\]

Тепер до точки перетину двох нерівностей можна наближатись ітеративно:

\begin{enumerate}
 \item припустимо деяке щонайменш можливе початкове значення ${x \to -\infty}$;
 \item з системи нерівностей, розв'язаної відносно $y$ отримаємо найбільше граничне значення $y$ з заданою точністю;
 \item змінюємо $y$, аналогічно знаходимо найбільше можливе значення $x$;
 \item переходимо до другого кроку;
 \item якщо на деякому кроці зміна $x$ і $y$ стала меншою за $\varepsilon$, то це число можна вважати точністю, з якою обчислений розв'язок.
\end{enumerate}

Попрацювавши над цією пробемою, читач може отримати більш універсальний алгоритм, а також узагальнити його для багатовимірних просторів.

\subsection*{Проблема №\arabic{chapter}.\ref{problem:linear:graphical_integer}}

Разом з ОДР на графіку можна також позначити всі точки $\langle x_1, x_2 \rangle \in \mathbb{Z}^2$ -- ті, які лежатимуть всередині ОДР і будуть допустимими розв'язками. Тоді оптимальний розв'язок слід обирати не з однієї з кутових точок, а з тих, що нанесені на графік. Приклад показано на Рис.~\ref{pic:linear:problem:integer_lp}

\illustration
 {../imgs/linear/integer-lp.pdf}
 {Чорними крапками позначено всі допустимі розв'язки з ОДР.}
 {pic:linear:problem:integer_lp}

У випадку, якщо лише одна з координат повинна бути цілочисельною, то на графіку отримаємо набір горизонтальних або вертикальних ліній.

\subsection*{Проблема №\arabic{chapter}.\ref{problem:linear:dim_reduction}}

\textit{Для розв'язування цієї проблеми читач повинен мати хороше інтуїтивне розуміння принципу вимірності простору.} Якщо ми розв'язуємо тривимірну задачу, то замість того, щоби малювати просторовий многогранник, можна намалювати на площині лише одну з його граней -- ту, ну якій за припущенням міститься розв'язок. Таким чином ми перетворюємо тривимірну проблему у двовимірну. Аналогічно можна аналізувати тривимірні перерізи чотиривимірних фігур тощо. Насправді така техніка немає важливого практичного сенсу, але може допомогти при унаочненні даних.

\subsection*{Проблема №\arabic{chapter}.\ref{problem:linear:infite_optimum}}

Якщо весь відрізок визначає оптимальні розв'язки, то це означає, що всі точки на ньому мають однакове значення $\varphi$ -- включно з двома крайніми. Отже, можна стверджувати, що якщо будь-які дві сусідні точки фігури мають однакове значення оптимальності, то всі точки з відрізку між ними є розв'язками задачі -- тобто, їх безліч.

У тривимірному просторі оптимальними розв'язками може бути вся грань фігури ОДР. Для того, щоби визначити такий випадок, потрібно знайти таку послідовність точок $\mathbf{X}_1 \ldots \mathbf{X}_n$, що вони всі є послідовно сусідніми та мають однакову оптимальність. Аналогічно у чотиривимірному просторі потрібно відшукати такий набір площин, що всі їхні кутові точки мають однакову оптимальність, і утворюють межі якогось багатогранника.

\subsection*{Проблема №\arabic{chapter}.\ref{problem:linear:convex_statement}}

Розглянемо матрично-векторний запис системи рівностей з люзовими змінними:

\[
 \mathbf{A} \mathbf{X}^\top + \mathbf{\delta} \mathbf{s}^\top = \mathbf{b}^\top\text{.}
\]

Якщо $\mathbf{\delta}$ -- одинична матриця, то рівняння перетворюється в таке:


\[
 \mathbf{A} \mathbf{X}^\top +
 \left[
  \begin{array}{cc}
   1 & 0 \\
   0 & 1
  \end{array}
 \right]
 \left[
  \begin{array}{c}
   s_1 \\
   s_2
  \end{array}
 \right] = \mathbf{b}^\top\text{.}
\]

Це означає, що всі $s_i$ входять у систему рівнянь зі знаком \flqq{}$+$\frqq{} і лише один раз. Як відомо додатні люзові змінні можуть з'явитись лише внаслідок перетворення нерівностей зі знаком \flqq{}$\leq$\frqq{}. З Рис.~\ref{pic:linear:problem:cannonical-statement} видно, що якщо для такої задачі створити додаткове обмеження $\forall i: x_i \leq 0$, то ОДР завжди буде опуклою та включатиме центр координат. Якщо окрім цього, якщо умова задачі вимагає максимізації цінової функції, і ОДР не порожня, то хоча б одна з координат вектора розв'язку $\mathbf{X}$ буде ненульовою.

\illustration
 {../imgs/linear/canonical-statement.pdf}
 {Під яким кутом не була би нахилена межа нерівності \flqq{}$\leq$\frqq{}, її область завжди буде спрямована до центру координат.}
 {pic:linear:problem:cannonical-statement}

\subsection*{Проблема №\arabic{chapter}.\ref{problem:linear:new_slack}}

\end{document}
