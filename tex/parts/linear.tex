\makeatletter
\def\input@path{{../}}
\makeatother

\providecommand{\main}{..}
\documentclass[\main/book.tex]{subfiles}

\begin{document}

\chapter{Лінійне програмуванння}

Одночасно простим для розуміння та корисним на практиці є математичний апарат розв'язку задач лінійного програмування (ЛП), з яким читач познайомиться у цьому розділі.

\begin{itemize}
 \item У параграфі \sectionref{section:linear:typical} читач дізнається про деякі типові проблеми ЛП та отримає інтуїтивне розуміння того, як варто формулювати ОДР і цінову функцію.
 \item У параграфі \sectionref{section:linear:geometrical} описано графічний метод розв'язку проблем ЛП для $\mathbf{X} \in \mathbb{R}^2$.
 \item Параграф \sectionref{section:linear:simplex} описує симплекс-метод розв'язування ЛП для $n$-вимірних проблем ($\mathbf{X} \in \mathbb{R}^{\geq 2}$).
 \item У параграфі \sectionref{section:linear:integer} читач дізнається про принцип границь та галужень для розв'язування цілочисельної задачі ЛП.
 \item У параграфі \sectionref{section:linear:x_in_0-1} описано постановку та розв'язування ЛП задачі при $\mathbf{X} \in \{0,1\}^n$.
\end{itemize}

Задачу можна назвати лінійною тоді, коли цінова функція є функцією першого степеня~(\ref{eq:p2:linear-phi}), а область допустимих значень може бути описана лінійними рівностями або нерівностями~(\ref{eq:p2:linear-g-set}).

\begin{equation}
\begin{split}
 \varphi(\mathbf{X}) =
 \sum_{j=1}^n c_j x_j =
 \mathbf{c} \mathbf{X}^\mathrm{T},\;
 \mathbf{c} = \langle c_1, \ldots, c_n \rangle
\end{split}
\label{eq:p2:linear-phi}
\end{equation}

\begin{equation}
 G = \left\{
  \langle x_1, \ldots, x_n \rangle:
  \sum_{j=1}^n a_{ij} x_j\,
   \boxed{\leq, \geq \text{або} =}\,
  b_i
  \,\forall i = \overline{1, m}
 \right\}
 \label{eq:p2:linear-g-set}
\end{equation}

\section{Типові проблеми}
\label{section:linear:typical}

Приведені далі постановки задач є типовими. Вони отримали свою назву від проблеми, у вирішенні якої найкраще використовувати такі означення. Якщо їх проаналізувати, то можна отримати відпрацьований математичний апарат для розв'язання більш специфічних задач.

\subsection{Максимізація прибутку}
\label{section:linear:phi_to_max}

\textit{Фабрика виробляє продукцію $n$ видів. Для неї потрібна певна кількість сировини $m$ видів, при чому її є в обмеженій кількості. Відомо можливий прибуток від реалізації кожного з видів продукції. Якщо існує якийсь оптимальний розподіл всієї сировини на виробництво певних видів продукції так, щоби отримати якнайбільший прибуток, то яким буде такий розподіл?}

Якщо між двома некерованими змінними існує відношення, то доцільно описати його у вигляді матриці. Тому нехай в $\mathbf{A}$ міститимуться елементи $a_{ij}$, кожен з яких означатиме витрату сировини $i$-ого виду на виготовлення продукції $j$-ого виду. Ще одна некерована змінна $\mathbf{b} = \langle b_1, \ldots, b_m \rangle$ позначатиме кількість відповідної сировини кожного виду, доступної на фабриці. Прибуток від реалізації можна визначити вектором $\mathbf{c} = \langle c_1, \ldots, c_n \rangle$.

Керованою змінною є вектор $\mathbf{X} = \langle x_1, \ldots, x_n \rangle$, кожен елемент якого позначає кількість продукції, яку має виготовити фабрика -- ці дані варто змінювати для досягнення оптимальності.

Тоді умовою допустимості розв'язку буде перш за все виконання обмежень на доступну кількість сировини, та деякі очевидні твердження -- наприклад, те, що ця кількість має бути невід'ємною (\ref{eq:p2:problem:phi_to_max:g_set}).

\begin{equation}
 \mathbf{X} \in G \Leftarrow
 \left\{
  \begin{array}{l}
   \displaystyle \sum_{j=1}^n a_{ij} x_j \leq b_i\;
   \forall i = \overline{1, m} \\
   x_j \geq 0 \; \forall j = \overline{1, n}
  \end{array}
 \right\}
 \label{eq:p2:problem:phi_to_max:g_set}
\end{equation}

\begin{question}
Чи можна поставити умову так, щоби знайдені $x_i$ обов'язково повинні були би бути цілими числами?
\end{question}

В означенні (\ref{eq:p2:problem:phi_to_max:g_set}) фор\-му\-лю\-ва\-ння ${\sum_{j=1}^n a_{ij} x_j}$ позначає суму використаної сировини певного типу на виготовлення ресурсів всіх можливих видів. Відповідно \flqq{}$\leq b_i$\frqq{} позначає, що цією сировини можна використати не більше певної кількості. Умова $\forall i$ означає, що така нерівність повинна виконуватись для всіх видів сировини.

Цінову функцію потрібно вводити так, щоби вона визначала прибутковість прийнятого рішення. Тоді оптимальним розв'язком буде такий, що гарантує досягнення її максимуму (\ref{eq:p2:problem:phi_to_max:price}).

\begin{equation}
 \varphi(\mathbf{X}) =
 \sum_{j=1}^n c_j x_j \rightarrow
 \max_{\mathbf{X} \in G}
 \label{eq:p2:problem:phi_to_max:price}
\end{equation}

\begin{question}
 Якщо існує така задача, для якої потрібно вводити кілька цінових функцій, і оптимізовувати їх одночасно, то якою буде умова такої задачі? Чи існує якийсь універсальний спосіб поєднання кількох оцінок оптимальності в одну?
\end{question}

\subsection{Мінімізація витрат}

Розглянемо задачу, подібну тій, що поставлена у розділі \ref{section:linear:phi_to_max}. \textit{Фабрика отримує сировину $n$ видів, з якої виготовляє товар за $m$ різними тех\-но\-ло\-гі\-я\-ми. Кожна з технологій потребує певну комбінацію сировини у різних кількостях. Разом з тим, кожна з технологій спричиняє в атмосферу певну кількість кілограм вуглецю. Для кожного виду сировини визначена доступна кількість, яку можна використати для виробництва. Потрібно оптимізувати виробництво так, щоби отримати якомога більше готової продукції, але водночас зменшити кількість шкідливих викидів.}

Що в цій задачі уже відомого? Найперше -- відношення технології та сировини, яку вона потребує. Запишемо це в матрицю $\mathbf{A}$, в якій кожен елемент $a_{ij}$ позначатиме витрату сировини $i$-ого типу за $j$-ою технологією. Викиди різних технологій можна зберігати у векторі $\mathbf{c} = \langle c_1, \ldots, c_m \rangle$, а кількість доступної сировини кожного виду -- як $\mathbf{b} = \langle b_1, \ldots, b_n \rangle$.

Що потрібно знайти? Потрібно знайти кількість товару, яка буде виготовлена за кожною з технологій. Тому позначимо її через вектор $\mathbf{X} = \langle x_1, \ldots, x_m \rangle$. При пошуку значення різних $x_j$ доведеться вирішити дилему: користуватись технологіями, які потребують меншу кількість сировини, але спричиняють більше викидів, чи навпаки?

Будь-який розв'язок $\mathbf{X}$ буде допустимим, якщо задовольнятиме дві умови: використано ресурсів не більше, ніж доступно взагалі, і жоден з $x_j$ не є від'ємним. Тому ознака допустимості розв'язку є такою самою, як і (\ref{eq:p2:problem:phi_to_max:g_set}) -- ми використовуємо ті самі назви змінних, тому формула не втрачає свого змісту. Тепер формулювання ${\sum_{j=1}^n a_{ij} x_j}$ позначатиме кількість сировини певного виду, використаної загалом, за якими технологіями б не відбувалось виробництво.

Допоки розв'язок допустимий, він використовуватиме лише дос\-туп\-ну кількість сировини. Однак різні розв'язки все ще спричиняють різний викид вуглецю, топу оптимальтність кожного з них визначатимемо саме за цим параметром. Цінова функція знову така сама, як і в (\ref{eq:p2:problem:phi_to_max:price}), проте цього разу її доцільно мінімізувати.

\begin{equation}
 \varphi(\mathbf{X}) =
 \mathbf{c} \mathbf{X}^\mathrm{T}
 \rightarrow \min_{\mathbf{X} \in G}
\end{equation}

\begin{note}
Тепер, після ознайомленнями з двома задачами, важливо розуміти, що не існує жодних правил, як варто складати формулювання ОДР чи цінової функції, так само як і неможливо передбачити всі можливі \flqq{}типові задачі\frqq{}. Формулювання будь-якої математичної моделі залежить винятково від області дослідження та винахідливості дослідника.
\end{note}

\subsection{План перевезень}

Досі ми розглядали задачі, де невідомою керованою змінною був вектор -- тобто, кількість чогось різних видів, або оптимальна конфігурація різних характеристик. Чи можливо тепер за допомогою лінійного програмування розв'язати якусь проблему, що встановлюватиме конфігурацію відношення? Тобто, цього разу невідоме $\mathbf{X}$ буде двовимірним -- матрицею.

\textit{Компанія має у своєму розпорядженні $n$ зерносховищ та $m$ постачальників зерна. Кожне зерносховище може зберігати певну кількість тон зерна, купленого у різних постачальників. З ними укладено контракт, за яким компанія зобов'язується викупити щонайменш певну кількість тон за раз. Відомі затрати на перевезення зерна від кожного постачальника до кожного сховища -- скажімо, час, відстань, пальне тощо. Потрібно визначити, скільки зерна між різними постачальниками та відповідними сховищами доцільно перевезти, щоби отримати якомога менші витрати, та перевезти якомога більше вантажу.}

\begin{figure}[!ht]
 \center
 \includegraphics{linear/transport-graph.pdf}
 \caption{Кожен з постачальників пропонує $\mathbf{a} = {\langle a_1, \ldots, a_m \rangle}$ тон, а кожне зі сховищ може зберігати до $\mathbf{b} = {\langle b_1, \ldots, b_n \rangle}$. Затрати на перевезення між ними задаються матрицею $\mathbf{C} \in \mathbb{R}^{m \times n}$.}
 \label{pic:linear:transport_problem}
\end{figure}

Отже, умову проблеми можна зобразити у вигляді графа, що на Рис.~\ref{pic:linear:transport_problem}. Зрозуміло, що некерованими змінними є $\mathbf{a}$, $\mathbf{b}$ та $\mathbf{C}$. Тоді розв'язком буде матриця $\mathbf{X}$, кожен елемент $x_{ij}$ якої задаватиме \flqq{}оптимальну\frqq{} кількість тон, яку варто перевезти між $i$-им постачальником та $j$-им сховищем. Саме така конфігурація, що буде задана цією матрицею і гарантуватиме найменші витрати та якнайбільше заповнення сховищ.

\begin{note}
 Ця проблема є класичним випадком транспортної задачі, і для її розв'язку існують набагато ефективніші методи, про які можна дізнатись у наступних розділах. Попри це, розв'язок саме лінійним програмуванням є досить цікавим.
\end{note}

Тепер ОДР повинно мати такі обмеження:

\begin{itemize}
 \item потрібно вивезти весь товар, що надають постачальники;
 \item до кожного з сховищ не можна привезти більше, ніж воно може вмістити.
\end{itemize}

Тоді умову допустимості розв'язку можна записати як \ref{eq:linear:problem:matrix_to_matrix:g_set}.

\begin{equation}
 \mathbf{X} \in G \Leftarrow \left\{
 \begin{array}{l}
  \displaystyle
  \sum_{i=1}^m x_{ij} = a_i\; \forall j = \overline{1, n} \\
  \displaystyle
  \sum_{j=1}^n x_{ij} \leq b_j\; \forall i = \overline{1, m}
 \end{array}
 \right\}
 \label{eq:linear:problem:matrix_to_matrix:g_set}
\end{equation}

Тепер для того, щоби порахувати затрати при поточному плані перевезення, достатньо перемножити значення $c_{ij}$ з матриці затрат на значення $x_{ij}$. Оптимальним буде той розв'язок, у якому це значення найменше, тому вводимо цінову функцію (\ref{eq:linear:problem:matrix_to_matrix:price}), і вимагаємо її мінімізації.

\begin{equation}
 \varphi(\mathbf{X}) =
 \sum_{i=1}^m \sum_{j=1}^n
 c_{ij} x_{ij}
 \rightarrow \min_{\mathbf{X} \in G}
 \label{eq:linear:problem:matrix_to_matrix:price}
\end{equation}

\begin{question}
 Нехай матриця $\mathbf{C}$ відображає не затрати на перевезення, а максимальну пропускну здатність маршруту. Скажімо, в день по дорозі між двома пунктами можна перевезти не більше, ніж $c_{ij}$ тон вантажу. Потрібно знайти таке завантаження $\mathbf{X}$ різних доріг, щоби перевезти якомога більше вантажу. Допускається ситуація, коли не весь вантаж буде вивезено від постачальників. Як сформулювати ОДР та цінову функцію?
\end{question}

\section{Геометричний сенс}
\label{section:linear:geometrical}

\section{Симплекс-метод}
\label{section:linear:simplex}

\section{Цілочисельне лінійне програмування}
\label{section:linear:integer}

\section{Проблеми оптимального вибору}
\label{section:linear:x_in_0-1}

Припустимо таку ситуацію, коли ми маємо набір різних рішень, з яких складається розв'язок проблеми. Наприклад, \textit{складне обчислення можна розпаралелити на $n$ процесорів, кожен з яких має свою продуктивність операцій/с та потужність у ватах. Скільки і які процесори потрібно увімкнути, щоби продуктивності вистачило на виконання деякого обчислення у вказаний час, але спожити щонайменше енергії?} Тоді набір всіх можливих рішень

\[
\begin{split}
P = \{
 \text{увімкнути процесор 1},
 \text{увімкнути процесор 2},
 \ldots, \\
 \text{увімкнути процесор $n$}
\}
\end{split}
.
\]

Відповідно, будь-який розв'язок задачі буде лише підмножиною $P$. Однак такими формулюваннями оперувати дуже не дуже зручно, тому краще ввести характеристичний вектор $\mathbf{X}$, кожен елемент $x_i$ якого позначатиме, чи прийняте деяке $i$-те можливе рішення. Так, якщо $x_i=0$, то $i$-ий процесор варто вимкнути, але якщо $x_i=1$ -- то навпаки.

\begin{note}
 Формально такий випадок можна записати як $\mathbf{X} \in \{0, 1\}^{n}$, або простіше -- $\forall i: x_i \in \{0, 1\}$.
\end{note}

\end{document}
