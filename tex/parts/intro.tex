\makeatletter
\def\input@path{{../}}
\makeatother
\documentclass[../book.tex]{subfiles}

\begin{document}

\nonumchapter{Введення}

Говорячи про \flqq{}дослідження операції\frqq{} ми маємо на увазі знаходження оптимального в якомусь сенсі розв'язку заданої математичної задачі. Задля знаходження розв'язку потрібно побудувати математичну модель. Як це зробити?

\begin{itemize}

 \item
 
 Спершу виділяємо \textbf{параметри операції} -- це ті дані, що подаються як факт, на них неможливо вплинути. Тому доцільно називати їх ще \textbf{некерованими змінними} чи константами.
 
 Наприклад, якщо \textit{склад місткістю в $m$ тон заповнений на $n$ тон кавунами}, то $m$ та $n$ є некерованими змінними.

 \item
 
 Опісля потрібно з'ясувати, на які змінні ми все таки можемо вплинути. Вони називаються \textbf{керувальними параметрами} або ж \textbf{керованими змінними}. Присвоєння конкретних значень керованим змінним і є розв'язком задачі.
 
 Наприклад, якщо потрібно з'ясувати, \textit{скільки $k$ тон кавунів ще можна довезти додатково}, то $k$ і є тією змінною, на яку можна і потрібно впливати, щоби забезпечити виконання потрібних умов.
 
 \item
 
 Обов'язково потрібно визначити деяку множину $G$, в якій міститимуться усі можливі допустимі розв'язки, або так звану \textbf{область допустимих розв'язків} (ОДР). Тоді, якщо $k \notin G$, то такий розв'язок не є допустимим і не приймається.
 
 Якщо врахувати всі умови, наведені вище, то доречно визначити ОДР як $$ G = \{{k \in \mathbb{R}}:  {0 \leq k \leq m-n}\}. $$

 Тобто, на склад можна довезти таку невід'ємну $k$ кількість тон, щоби вона обов'язково помістилась на складі. $(m-n)$ -- вільне місце на складі до виконання операції.

 \item
 
 І наостанок, яким чином обрати найоптимальніший розв'язок з множини усіх допустимих? Для цього прийнято вводити \textbf{цінову функцію}, що позначала би оптимальність (якість) будь-якого допустимого розв'язку чисельно. Тоді задача пошуку розв'язку зводиться до пошуку екстремуму цієї функції.
 
 Наприклад, якщо потрібно з'ясувати найбільше можливе $k$, то цінову функцію можна ввести як $$ \varphi(k) = m-n-k, $$ тобто вона визначала би вільне місце, яке залишається після того, як довезти ще $k$ тон.
 
 \item
 
 Тепер задачу можна сформулювати так: \textit{знайти таке $k$, щоби виконувались умови $\varphi(k) \rightarrow \max$ та $k \in G$}. Завдяки тому, що цінова функція введена саме так, вона визначатиме найбільш оптимальними ті $k$, які сприяють тому, щоби на складі залишалось якомога менше вільного місця. Шукаючи максимум цієї функції, ми одночасно знаходимо максимально можливе і допустиме $k$.

\end{itemize}

Подана задача, ймовірно, видається читачу занадто примітивною. Все правильно, для її розв'язку не обов'язково виконувати жодного оптимізаційного процесу та користуватись складними визначеннями. Відповідь можна знайти з простого рівняння: $k = m - n$. Якщо нам вдається його скласти, і знайти розв'язок за один крок, це найкраще з того, що могло статись. Тоді математична модель такої задачі називається \textbf{аналітичною}. Такий приклад подано лише для того, щоби продемонструвати важливі базові поняття на простих умовах.

Важливо розуміти, що іноді \flqq{}знайти розв'язок\frqq{} означає знаходження значень десятків керованих змінних, які якось впливають одна на одну, і в сукупності можуть змінювати свою оптимальність непередбачуваним і неочевидним чином. Пошук розв'язку може нагадувати намагання збалансувати складну систему тягарців: як тільки потягнути за один, одразу ж починає рух інший. Відповідь на такі задачі найчастіше не може бути знайдена за один крок, і математична модель такої задачі є \textbf{алгоритмічною} (навіть коли алгоритм завершується одразу після першого кроку). Тоді й стає у пригоді описаний математичний апарат. Далі подано приклади кількох задач, які можна розв'язати алгоритмічно.

\begin{itemize}
 \item Відома відстань між будь-якими двома складами $a_i$ та $a_j$ і заробіток $b_i$ від доставки на $i$-ий склад всього товару, який на ньому замовили. Потрібно скласти маршрут від $a_1$ до $a_n$, щоби пройти якомога меншу відстань, і заробити якомога більше грошей.
 \item Попередню умову можна ускладнити, припустивши, що існує три типи товарів, і вартість кожного з них встановлюють функції $f_1(x_1)$, $f_2(x_2)$ та $f_3(x_3)$, де $x_i$ -- кількість $i$-ого товару в певній одиниці виміру.
 \item Існує $i$ типів речей, і в кожної з них є своя маса $m_i$ кілограм та вартість $c_i$. Скільки і яких речей потрібно покласти в рюкзак, щоби сумарно отримати найбільшу можливу вартість, але набрати не більше ніж $M$ кілограм в цілому?
 \item $i$-ий робітник на $j$-ому виді діяльності має продуктивність $c_{ij}$. Як розподілити робітників по різних роботах так, щоби сумарно отримати найбільшу продуктивність?
\end{itemize}

Умови всіх наведених задач формують \textbf{детерміновані} математичні моделі, тобто значення некерованих змінних та можливої цінової функції завжди залишається незмінним. Однак може бути так, що всі параметри можна передбачити заздалегідь тільки з певною вірогідністю. Така ситуація може бути, наприклад, на: грошовій біржі; ринку товарів, де сьогоднішню ціну визначає низка складних характеристик, а іноді -- звичайний людський непередбачуваний вибір; при прогнозуванні погодних умов тощо. Тоді математична модель такої задачі називається \textbf{стохастичною}.

Цей посібник надає детальні інструкції до побудови математичної моделі найпоширеніших операцій та методи знаходження їхніх розв'язків.

\end{document}
