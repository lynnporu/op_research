\makeatletter
\def\input@path{{../}}
\makeatother
\documentclass[../book.tex]{subfiles}

\begin{document}

\chapter{Введення}

Говорячи про \flqq{}дослідження операції\frqq{}, ми маємо на увазі пошук рішення якоїсь проблеми, пов'язаної із реальним життям. Дослідження полягає в тому, щоби знайти якнайкращий спосіб представлення реального світу чисельно: у вигляді матриць, векторів або функціональних залежностей. Тоді, використовуючи математичний апарат, відповідь на задачу можна знайти із будь-якою точністю. Після цього її потрібно інтерпретувати вже у \flqq{}людських\frqq{} поняттях.

Мистецтво чисельного опису світу називається побудовою \textbf{математичної моделі}. Така потреба виникала настільки часто, що з отриманих результатів можна виділити загальні підходи та ідеї, і використовувати їх на інших, ще не досліджених проблемах (операціях).

Цей посібник надає приклади багатьох популярних задач, що можуть бути розв'язані з допомогою лінійного, нелінійного або динамічного програмування, а також аналіз теорії, що стоїть за ними. Автор сподівається, що це дозволить сформувати хороше розуміння дослідження операцій, а також набути інтуїцію, яку потім можна використовувати при розв'язуванні нових, невідомих досі проблем.

\section{Побудова математичної моделі}

Логічно, що з початком аналізу будь-якої задачі, варто виділити ті дані, які подаються як факт -- тобто, на них неможливо вплинути, чи змінити будь-як. Це те, біля чого можна написати \flqq{}дано\frqq{}, і далі подати перелік змінних з присвоєними їм значеннями. Вони називаються \textbf{параметрами операції}, або \textbf{некерованими змінними}.

Якщо якісь дані описують одну і ту саму характеристику різних предметів, цілком доцільно об'єднати їх у вектор. Надалі використовуватиметься така нотація: $\mathbf{a} = \langle a_1, \ldots, a_n \rangle^\top$. В тих випадках, коли подані дані є двовимірними, чи відображають деякі залежності, можна використовувати матриці. Наприклад, якщо в умові сказано, що \textit{відомо час перевезення вантажу між кожним із обласних центрів}, то їх можна пронумерувати, а дані помістити, скажімо, в матрицю $\mathbf{T}$, де кожен елемент $t_{ij}$ позначатиме час перевезення з $i$-ого міста в $j$-те.

Опісля маємо з'ясувати, які ж дані в задачі залишились невідомими, і їх потрібно знайти -- себто, визначити на основі відомих. Вони називаються \textbf{керувальними параметрами}, або ж \textbf{керованими змінними}. Присвоєння цим змінним конкретних значень і є вирішенням проблеми.

У будь-яких задачах, які читач міг бачити до цього, зазвичай по\-тріб\-но було з'ясувати значення одного невідомого параметра (деякого $x$), або навіть кількох. При дослідженні операцій може так траплятись, що невідомою буде ціла матриця, або вектор.

Якщо існує таке рівняння, що явно пов'язувало б некеровані змінні з керованими, то розв'язок можна знайти \textbf{аналітично}, тобто, обчислити звичайними арифметичними операціями. Наприклад, якщо відомо, що \textit{пропускна спроможність каналу передачі становить $n$ Мб/с, і потрібно з'ясувати, скільки максимально $x$ Мб інформації можна передати за $t$ секунд}, то зрозуміло, що відповідь може бути знайдена з формули ${x=tn}$.

Однак бувають такі проблеми, для яких неможливо -- або принаймні дуже складно -- скласти будь-яке рівняння, тому що у них взаємодіє між собою велика кількість змінних, кожна з яких по-своєму впливає на оптимальність результату в цілому. Тоді розв'язування може нагадувати намагання збалансувати складну систему тягарців: як тільки потягнути за один, одразу ж починають рух інші. Відповідь на такі проблеми найчастіше можна знайти \textbf{алгоритмічно}, на кожному з кроків покращуючи оптимальність результату. Власне, саме такий спосіб і використовується найчастіше для дослідження операцій.

Для того, щоби розв'язати задачу алгоритмічно, прийнятою практикою є побудова (опис) деякої множини $G$, що містить всі можливі допустимі розв'язки задачі. Звичайно, перед цим потрібно визначити, в якому вигляді ми взагалі шукатимемо розв'зок. Скажімо, якщо треба знайти дві змінні $x_1, x_2 \in \mathbb{R}$, то розв'язком буде вектор $\mathbf{X} = {\langle x_1, x_2 \rangle^\top}$, і множина розв'язку -- $G \subset \mathbb{R}^2$. Іноді рішенням проблеми може бути матриця, чи складніші багатовимірні об'єкти.

Наостанок, для того, щоб оцінити оптимальність будь-якого роз\-в'яз\-ку, прийнято вводити \textbf{цінову функцію} (інша назва -- \textbf{коефіцієнт ефективності}), що встановлює чисельну оцінку якості ${\varphi: G \rightarrow \mathbb{R}}$. Формулювання функції обирається залежно від умови.

Наприклад, якщо невідомі змінні $x_1$, $x_2$ встановлюють кількість товару двох видів, яку виготовлятиме завод, а $c_1$, $c_2$ -- їхню ціну відповідно, то коефіцієнт ефективності можна визначити як прибуток від всього товару загалом: ${\varphi(\mathbf{X})} = {c_1 x_1} + {c_2 x_2} = \mathbf{c}^\top \mathbf{X}$. Тоді знайти такий $\mathbf{X}$, що забезпечував би максимальний прибуток, означає знайти екстремум $\varphi$.

\section{Класифікація моделей}

Запропонована структура математичної моделі дозволяє описати будь-яку операцію чисельно, зводячи розв'язування до пошуку такого ${\mathbf{X} \in G}$, щоби досягнути мінімуму або максимуму $\varphi$. Зрозуміло, що роз\-в'яз\-ува\-ти такі проблеми перебором, щонайменш нефеективно. На щастя, для цього існують спеціальні методи, проте вони гарантують знаходження адекватного розв'язку лише для якогось конкретного типу математичної моделі. Тому важливо розрізняти, які типи існують взагалі, і вміти визначити їх для моделі своєї задачі. Далі приведено класифікацію математичних моделей за характерною ознакою.

\begin{table}[!ht]
\centering
\begin{tabular}{p{5.5cm}|ll}
 & Так & Ні \\
 Рішення проблеми знаходиться розв'язком рівняння. & \textbf{аналітична} & \textbf{алгоритмічна} \\
 Хоча б одна з величин є випадковою величиною. & \textbf{стохастична} & \textbf{детермінована} \\
 Кожен з кроків алгоритму розв'язку покращує рішення, знайдене на попередньому. & \textbf{статична} & \\
 На кожному з кроків алгоритму розв'язку приймається найоптимальніше рішення. & & \textbf{динамічна} \\
 Хоча б одна з керованих змінних може приймати тільки цілі значення. & \textbf{дискретна} & \textbf{недискретна}
\end{tabular}
\end{table}

\end{document}
