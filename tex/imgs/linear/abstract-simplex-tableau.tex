\documentclass{standalone}

\usepackage{tikz}

\usepackage[T2A]{fontenc}
\usepackage[utf8]{inputenc}
\usepackage[english, ukrainian]{babel}

\usepackage[default,scale=.90]{opensans}
\usepackage{newtxmath}

\renewcommand {\baselinestretch} {1.5}
\setlength\parindent{1cm}

% creates numerated cell
\newcounter{nodecount}
\newcommand\numcell[1]{
    \addtocounter{nodecount}{1}
    \tikz \node (\arabic{nodecount}) {#1};
}

% creates named cell
\newcommand\namecell[2]{
    \tikz \node (#1) {#2};
}

% creates label and points to the box in table
% #1 - parameters of lable (such as placement or smth)
% #2 - parameters of arrow
% #3 - point to this point's name
% #4 - set name of label's node
% #5 - text
% #6 - color
\newcommand\explain[6]{
    \node [explanation, fill=#6!40!white, #1]
        at (#3) (#4) {#5};
    \draw [
        thick, #2, #6
    ] (#3) to (#4);
}

% creates box with one row height
% #1 - color
% #2 - first node, #3 - second node
\newcommand\vboxontwo[3]{
 \draw [box, #1]
    (#2.north west) -- (#3.north east) --
    (#3.south east) -- (#2.south west) --
    cycle;
}

% box with one column width
\newcommand\hboxontwo[3]{
 \draw [box, #1]
    (#2.north west) -- (#2.north east) --
    (#3.south east) -- (#3.south west) --
    cycle;
}

\newcommand\singlebox[2]{
 \draw [box, #1]
    (#2.north west) -- (#2.north east) --
    (#2.south east) -- (#2.south west) --
    cycle;
}

\definecolor{palegreen}   {rgb}{0.6,  0.98, 0.6}
\definecolor{pastelorange}{rgb}{1.0,  0.7,  0.28}
\definecolor{pastelred}   {rgb}{1.0,  0.41, 0.38}
\definecolor{pastelblue}  {rgb}{0.68, 0.78, 0.81}
\definecolor{palegreen}   {rgb}{0.6,  0.98, 0.6}
\definecolor{pinkorange}  {rgb}{1.0,  0.6,  0.4}
\definecolor{magicmint}   {rgb}{0.67, 0.94, 0.82}
\definecolor{darkcyan}    {rgb}{0.0,  0.72, 0.92}
\definecolor{cream}       {rgb}{1.0,  0.99, 0.82}
\definecolor{lavender}    {rgb}{0.71, 0.49, 0.86}
\definecolor{mintgreen}   {rgb}{0.6,  1.0,  0.6}

\begin{document}

\tikzstyle{every picture}+=[remember picture,baseline]
\tikzstyle{every node}+=[anchor=base, outer sep=4pt, inner sep=0]

\begin{minipage}[t][91mm][c]{113mm}
 \setlength{\tabcolsep}{5pt}
 \centering
  \begin{tabular}{rlllll}
    \multicolumn{1}{l}{}       & $x_1$    & $x_2$    & $s_1$ & $s_2$                  & \textbf{План} \\ \cline{2-5}
    \namecell{basis_cell1}{$s_1$} & \multicolumn{1}{|r}{$a_{11}$} & $a_{12}$ & $1$   & \multicolumn{1}{l|}{$0$} & \namecell{bound_cell1}{$b_1$}         \\
    \namecell{basis_cell2}{$s_2$} & \multicolumn{1}{|r}{\namecell{a21_cell}{$a_{21}$}} & $a_{22}$ & $0$   & \multicolumn{1}{l|}{\namecell{s2_cell}{$1$}} & \namecell{bound_cell2}{$b_2$}         \\ \cline{2-5}
    $\varphi$                  & $c_1$    & $c_2$    & \namecell{price0_cell0}{$0$}   & \namecell{price0_cell1}{$0$}                    &
  \end{tabular}
\end{minipage}
   
\begin{tikzpicture}[
    overlay,
    box/.style={thick, rounded corners},
    explanation/.style={
        align=left,
        text width=5cm,
        inner sep=5pt
    }
]

 % paths

 \hboxontwo{pinkorange}   {bound_cell1}     {bound_cell2};
 \hboxontwo{darkcyan}     {basis_cell1}     {basis_cell2};
 \vboxontwo{pastelred}    {price0_cell0}    {price0_cell1};
 \singlebox{lavender}     {a21_cell};
 \singlebox{palegreen}    {s2_cell};
    
 % labels

 \explain
    {right=-1.8cm, above=-3.5cm} {in=50, out=-130}
    {a21_cell}{a21_box}
    {
        Змінна $x_1$ входить з коефіцієнтом $a_{21}$ у рівняння зі змінною симплекс-базису $s_2$.
    }
    {lavender};

 \explain
    {right=-2.1cm, above=-1cm, text width=2.7cm} {in=20, out=90}
    {basis_cell1}{basis_box}
    {
        Змінні симплекс-базису, з якими асоціюються відповідні рівняння.
    }
    {darkcyan};

 \explain
    {right=-1cm, above=2cm, text width=3.5cm} {in=-80, out=120}
    {s2_cell}{s2_box}
    {
        Змінні симплекс-базису завжди входять з коефіцієнтом $1$ у власні рівняння.
    }
    {palegreen};

 \explain
    {right=2.4cm, above=-1cm, text width=2.5cm} {}
    {bound_cell1}{bound_box}
    {
        Ці значення є одночасно значеннями симплекс-базису.
    }
    {pinkorange};


 \explain
    {right=1.7cm, above=-3cm, text width=3.8cm} {out=-90}
    {price0_cell0}{price_box}
    {
        Змінні симплекс-базису входять у цінову функцію з коефіцієнтом $0$.
    }
    {pastelred};

\end{tikzpicture}

\end{document} 
