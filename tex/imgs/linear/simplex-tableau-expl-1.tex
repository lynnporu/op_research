\documentclass{standalone}

\usepackage{tikz}
\usetikzlibrary{decorations.markings}

\usepackage[T2A]{fontenc}
\usepackage[utf8]{inputenc}
\usepackage[english, ukrainian]{babel}

\usepackage[default,scale=.90]{opensans}
\usepackage{newtxmath}

\renewcommand {\baselinestretch} {1.5}

\definecolor{palegreen}   {rgb}{0.6,  0.98, 0.6}
\definecolor{pastelorange}{rgb}{1.0,  0.7,  0.28}
\definecolor{pastelred}   {rgb}{1.0,  0.41, 0.38}
\definecolor{pastelblue}  {rgb}{0.68, 0.78, 0.81}
\definecolor{palegreen}   {rgb}{0.6,  0.98, 0.6}
\definecolor{pinkorange}  {rgb}{1.0,  0.6,  0.4}
\definecolor{magicmint}   {rgb}{0.67, 0.94, 0.82}
\definecolor{darkcyan}    {rgb}{0.0,  0.72, 0.92}
\definecolor{cream}       {rgb}{1.0,  0.99, 0.82}
\definecolor{lavender}    {rgb}{0.71, 0.49, 0.86}
\definecolor{mintgreen}   {rgb}{0.6,  1.0,  0.6}

\newcommand\explain[6]{
    \node [explanation, fill=#6!40!white, #1]
        at (#3) (#4) {#5};
    \draw [
        <-, thick, #2, #6
    ] (#3) to (#4);
}

\begin{document}
\begin{minipage}[c][100mm][c]{113mm}

\center
\begin{tikzpicture}[
    point/.style={circle, fill, inner sep=1.2pt, black},
    axis/.style={thin, ->},
    ticklabel/.style={
        fill=white
    },
    explanation/.style={
        align=left,
        text width=5cm,
        inner sep=5pt
    },
    scale=.65
]

    \draw[blue, fill=blue!20!white, opacity=.3]
        (0, 4) -- (2, 0) -- (0, 0) -- cycle;

    \draw[blue, fill=blue!20!white, opacity=.3]
        (0, 1.33) -- (4, 0) -- (0, 0) -- cycle;
    
    \draw[
        thin,
        dotted,
        decoration={
            markings,
            mark=at position 0.97 with {\arrow{>}}
        },
        postaction=decorate
    ] (0, 0) -- (3, 4.5)
        node [anchor=west, xshift= 5pt] {$
            \overrightarrow{\mathrm{grad}}\, \varphi
        $};
    
    \draw(0, 0)
        node (point_0_0) [point] {};
        
    \node at (0, 4) (point_0_4) {};
    
    \explain
        {right=-2.7cm, above=2cm, text width=3.2cm} {in=-35, out=0}
        {point_0_4}{point_0_4_box}
        {
            За таблицею, одна з меж, допоки можна збільшувати $x_2$.
        }
        {pinkorange}
    
	\draw[axis]
        (0,0) -- (0,5)
        node[ticklabel, anchor=south]{$x_2$};
	\draw[axis]
        (0,0) -- (4.5,0)
        node[ticklabel, anchor=west]{$x_1$};

    \node at (-.25, 2.8) (y_axis_point) {};
        
    \explain
        {right=-2.7cm, above=.5cm, text width=3cm} {in=-40, out=-90}
        {y_axis_point}{y_axis_box}
        {
            $x_1=0$, решта параметрів змінюються.
        }
        {pink}
        
    \node at (0, 1.33) (point_0_133) {};
    
    \explain
        {right=-2.9cm, above=-2.5cm, text width=3.4cm} {in=0, out=-45}
        {point_0_133}{point_0_133_box}
        {
            Найменша межа збільшення -- ${x_2 \leq 1.33}$. Якщо продовжувати збільшення, розв'язок вийде з ОДР. 
        }
        {pinkorange}
        
    \foreach \x in {1.33, 4, 3}{
        \draw(0, \x) node[point] {};
        \draw[thin]
            (0, \x) -- (-.15, \x)
            node[ticklabel, anchor=east] {$\x$};
            
    }
        
    \foreach \y in {2, 4, 2}{
        \draw(\y, 0) node[point] {};
        \draw[thin]
            (\y, 0) -- (\y, -.15)
            node[ticklabel, anchor=north] {$\y$};
    }
    
    \draw [dotted] (0, 3) -- (2, 3) -- (2, 0);
    
    \explain
        {right=3.8cm, above=.5cm, text width=3.2cm} {in=-180, out=45}
        {.2, .2}{point_0_0_box}
        {
            Поточний розв'язок, \\
            $\mathbf{X}_0 = \langle x_1, x_2, s_1, s_2 \rangle = \langle 0, 0, 4, 4 \rangle$
        }
        {lavender};
        
    \node at (3.9, -.25) (x_axis_point) {};
    
    \explain
        {right=.7cm, above=-2.5cm} {in=150, out=180}
        {x_axis_point}{x_axis_box}
        {
            Рух в цьому напрямку означає незмінний $x_2=0$, але змінні решту параметрів.
        }
        {pink};
        
    \node at (1.66, 3.23) (point_166_323) {};
    \node at (0.45, 3.60) (point_045_360) {};
    
    \draw[
        thick, palegreen, ->,
        in=10, out=-220
    ] (point_166_323) to (point_045_360);
    
    \explain
        {right=1cm, above=3cm, text width=5cm} {-, in=-120, out=-40}
        {1.46, 3.40}{price_box}
        {
            Оскільки градієнт нахилений більше в сторону осі $x_2$, вигідніше збільшувати саме $x_2$.
        }
        {palegreen};
    
\end{tikzpicture}
\end{minipage}

\end{document} 
